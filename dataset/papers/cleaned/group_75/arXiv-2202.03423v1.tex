\title{Backdoor Defense via Decoupling the Training Process}

\begin{document}

\maketitle

\begin{abstract}
Recent studies have revealed that deep neural networks (DNNs) are vulnerable to backdoor attacks, where attackers embed hidden backdoors in the DNN model by poisoning a few training samples. The attacked model behaves normally on benign samples, whereas its prediction will be maliciously changed when the backdoor is activated. We reveal that poisoned samples tend to cluster together in the feature space of the attacked DNN model, which is mostly due to the end-to-end supervised training paradigm. Inspired by this observation, we propose a novel backdoor defense via decoupling the original end-to-end training process into three stages. Specifically, we first learn the backbone of a DNN model via \emph{self-supervised learning} based on training samples without their labels. The learned backbone will map samples with the same ground-truth label to similar locations in the feature space. Then, we freeze the parameters of the learned backbone and train the remaining fully connected layers via standard training with all (labeled) training samples. Lastly, to further alleviate side-effects of poisoned samples in the second stage, we remove labels of some `low-credible' samples determined based on the learned model and conduct a \emph{semi-supervised fine-tuning} of the whole model. Extensive experiments on multiple benchmark datasets and DNN models verify that the proposed defense is effective in reducing backdoor threats while preserving high accuracy in predicting benign samples. Our code is available at \url{https://github.com/SCLBD/DBD}.
\end{abstract}

\section{Introduction}
\label{sec:intro}

\comment{
\begin{figure}[ht]
    \centering
    \includegraphics[width=0.473\textwidth]{fig_intro.pdf}
    \caption{The t-SNE of poisoned samples in the hidden space generated by attacked DNNs trained in supervised and self-supervised manner. As shown in the figure, poisoned samples tend to cluster together of DNN (within the red box) trained with standard supervised learning, whereas lie closely to samples with their ground-truth label of the one trained with self-supervised learning. }
    \label{fig:intro}
    \vspace{-1em}
\end{figure}
}

Deep learning, especially deep neural networks (DNNs), has been widely adopted in many realms \citep{wang2020pillar,li2020short,wen20adaptive} for its high effectiveness. In general, the training of DNNs requires a large amount of training samples and computational resources. Accordingly, third-party resources ($e.g.$, third-party data or servers) are usually involved. While the opacity of the training process brings certain convenience, it also introduces new security threats.
 

Backdoor attack poses a new security threat to the training process of DNNs \citep{li2020backdoor}. It maliciously manipulates the prediction of the attacked DNNs by poisoning a few training samples. Specifically, backdoor attackers inject the \emph{backdoor trigger} ($i.e.$, a particular pattern) to some benign training images and change their labels with the attacker-specified \emph{target label}. The connection between the backdoor trigger and the target label will be learned by DNNs during the training process. In the inference process, the prediction of attacked DNNs will be changed to the target label when the trigger is present, whereas the attacked DNNs will behave normally on benign samples. As such, users are difficult to realize the existence of hidden backdoors and therefore this attack is a serious threat to the practical applications of DNNs.

In this paper, we first investigate backdoor attacks from the hidden feature space. Our preliminary experiments reveal that the backdoor is embedded in the feature space, $i.e.$, samples with the backdoor trigger (dubbed \emph{poisoned samples}) tend to cluster together in the feature space. We reveal that \emph{this phenomenon is mostly due to the end-to-end supervised training paradigm}. Specifically, the excessive learning capability allows DNNs to learn features about the backdoor trigger, while the DNNs can shrink the distance between poisoned samples in the feature space and connect the learned trigger-related features with the target label by the end-to-end supervised training. Based on this understanding, we propose to decouple the end-to-end training process for the backdoor defense. Specifically, we treat the DNNs as two disjoint parts, including a \emph{feature extractor} ($i.e.$, backbone) and a \emph{simple classifier} ($i.e.$, the remaining fully connected layers). We first learn the \emph{purified feature extractor} via \emph{self-supervised learning} \citep{kolesnikov2019revisiting,chen2020simple,jing2020self} with unlabeled training samples (obtained by removing their labels), and then learn the simple classifier via standard supervised training process based on the learned feature extractor and all training samples. The strong data augmentations involved in the self-supervised learning damage trigger patterns, making them unlearnable during representation learning; and the decoupling process further disconnects trigger patterns and the target label. Accordingly, hidden backdoors cannot be successfully created even the model is trained on the poisoned dataset based on our defense.

Moreover, we further reveal that the representation of poisoned samples generated by the purified extractor is significantly different from those generated by the extractor learned with standard training process. Specifically, the poisoned sample lies closely to samples with its ground-truth label instead of the target label. This phenomenon makes the training of the simple classifier similar to \emph{label-noise learning} \citep{wang2019symmetric,ma2020normalized,berthon2021confidence}. As such, we first filter \emph{high-credible training samples} ($i.e.$, training samples that are most probably to be benign) and then use those samples as labeled samples and the remaining part to form unlabeled samples to fine-tune the whole model via \emph{semi-supervised learning} \citep{rasmus2015semi,berthelot2019mixmatch,sohn2020fixmatch}. This approach is to further reduce the adverse effects of poisoned samples.

The main contributions of this paper are three-fold. \textbf{(1)} We reveal that the backdoor is embedded in the feature space, which is mostly due to the end-to-end supervised training paradigm. \textbf{(2)} Based on our understanding, we propose a decoupling-based backdoor defense (DBD) to alleviate the threat of poisoning-based backdoor attacks. \textbf{(3)} Experiments on classical benchmark datasets are conducted, which verify the effectiveness of our defense.

\vspace{-0.5em}
\section{Related Work}
\vspace{-0.5em}

\subsection{Backdoor Attack}
\label{sec:related_attack}

Backdoor attack is an emerging research area, which raises security concerns about training with third-party resources. In this paper, we focus on the poisoning-based backdoor attack towards image classification, where attackers can only modify the dataset instead of other training components ($e.g.$, training loss). This threat could also happen in other tasks \citep{xiang2021backdoor,zhai2021backdoor,li2022few} and with different attacker's capacities \citep{nguyen2020input,tang2020embarrassingly,zeng2021rethinking}, which are out-of-scope of this paper. 
In general, existing attacks can be divided into two main categories based on the property of target labels, as follows:

\noindent \textbf{Poison-Label Backdoor Attack.}
It is currently the most common attack paradigm, where the target label is different from the ground-truth label of poisoned samples. BadNets \citep{gu2017badnets} is the first and most representative poison-label attack. %Almost all follow-up attacks were carried out based on this method. 
Specifically, it randomly selected a few samples from the original benign dataset to generate \emph{poisoned samples} by stamping the \emph{backdoor trigger} onto the (benign) image and change their label with an attacker-specified \emph{target label}. Those generated poisoned samples associated with remaining benign ones were combined to form the \emph{poisoned training dataset}, which will be delivered to users. After that, \citep{chen2017targeted} suggested that the poisoned image should be similar to its benign version for the stealthiness, based on which they proposed the \emph{blended attack}. %In the blended attack, poisoned images were generated by blending the backdoor trigger with benign images instead of by stamping proposed in the BadNets. 
Recently, \citep{xue2020one,li2020invisible,li2021invisible} further explored how to conduct poison-label backdoor attacks more stealthily. Most recently, a more stealthy and effective attack, the WaNet \citep{nguyen2021wanet}, was proposed. WaNet adopted image warping as the backdoor trigger, which deforms but preserves the image content. 

\noindent \textbf{Clean-Label Backdoor Attack.} 
Although the poisoned image generated by poison-label attacks could be similar to its benign version, users may still notice the attack by examining the image-label relationship. To address this problem, \cite{turner2019label} proposed the \emph{clean-label attack paradigm}, where the target label is consistent with the ground-truth label of poisoned samples. Specifically, they first leveraged adversarial perturbations or generative models to modify some benign images from the target class and then conducted the standard trigger injection process. %The modification was to alleviate the effects of `robust features' about the target class contained in poisoned samples to ensure that the trigger can be successfully learned by the DNNs. 
This idea was generalized to attack video classification in \citep{zhao2020clean}, where they adopted the targeted universal adversarial perturbation \citep{moosavi2017universal} as the trigger pattern. %Recently, Liu \etal \cite{liu2020reflection} proposed to adopt reflection as the trigger pattern for the attack.
Although clean-label backdoor attacks are more stealthy compared with poison-label ones, they usually suffer from relatively poor performance and may even fail in creating backdoors \citep{li2020backdoor}.

\subsection{Backdoor Defense}
Currently, there are also some approaches to alleviate the backdoor threat. Existing defenses are mostly \emph{empirical}, which can be divided into five main categories, including \textbf{(1)} \emph{detection-based defenses} \citep{xu2021detecting,zeng2021rethinking,xiang2022post}, \textbf{(2)} \emph{preprocessing based defenses} \citep{doan2020februus,li2021backdoor2,zeng2021deepsweep}, \textbf{(3)} \emph{model reconstruction based defenses} \citep{zhao2020bridging,li2021neural,zeng2022adversarial}, \textbf{(4)} \emph{trigger synthesis based defenses} \citep{guo2020towards,dong2021black,shen2021backdoor}, and \textbf{(5)} \emph{poison suppression based defenses} \citep{du2019robust,borgnia2021strong}. Specifically, detection-based defenses examine whether a suspicious DNN or sample is attacked and it will deny the use of malicious objects; Preprocessing based methods intend to damage trigger patterns contained in attack samples to prevent backdoor activation by introducing a preprocessing module before feeding images into DNNs; Model reconstruction based ones aim at removing the hidden backdoor in DNNs by modifying models directly; The fourth type of defenses synthesize potential trigger patterns at first, following by the second stage that the hidden backdoor is eliminated by suppressing their effects; The last type of methods depress the effectiveness of poisoned samples during the training process to prevent the creation of hidden backdoors. In general, our method is most relevant to this type of defenses. %Besides, the third and fourth type of methods need a local benign dataset in most cases, which is not required in our method.

In this paper, we only focus on the last four types of defenses since they directly improve the robustness of DNNs. Besides, there were also few works focusing on \emph{certified backdoor defenses} \citep{wang2020certifying,weber2020rab}.
Their robustness is theoretically guaranteed under certain assumptions, which cause these methods to be generally weaker than empirical ones in practice.

\subsection{Semi-supervised and Self-supervised Learning}

\noindent \textbf{Semi-supervised Learning.} 
In many real-world applications, the acquisition of labeled data often relies on manual labeling, which is very expensive. In contrast, obtaining unlabeled samples is much easier. To utilize the power of unlabeled samples with labeled ones simultaneously, a great amount of semi-supervised learning methods were proposed \citep{gao2017semi,berthelot2019mixmatch,van2020survey}. Recently, semi-supervised learning was also introduced in improving the security of DNNs \citep{stanforth2019,carmon2019}, where they utilized unlabelled samples in the adversarial training. Most recently, \citep{yan2021dehib} discussed how to backdoor semi-supervised learning. However, this approach needs to control other training components ($e.g.$, training loss) in addition to modifying training samples and therefore is out-of-scope of this paper. How to adopt semi-supervised learning for backdoor defense remains blank.

\noindent \textbf{Self-supervised Learning.}
This learning paradigm is a subset of \emph{unsupervised learning}, where DNNs are trained with supervised signals generated from the data itself \citep{chen2020simple,grill2020bootstrap,liu2021self}. It has been adopted for increasing adversarial robustness \citep{hendrycks2019using,wu2021adversarial,shi2021online}. Most recently, there were also a few works \citep{saha2021backdoor,carlini2021poisoning,jia2021badencoder} exploring how to backdoor self-supervised learning. However, these attacks are out-of-scope of this paper since they need to control other training components ($e.g.$, training loss) in addition to modifying training samples.

\vspace{-0.5em}
\section{Revisiting Backdoor Attacks from the Hidden Feature Space}
\vspace{-0.5em}
\label{sec:motivation}

\begin{figure}[ht]
\centering
\vspace{-3em}
\subfigure[]{
\centering
\includegraphics[width=0.235\textwidth]{supervise_badnets_pratio0.05.pdf}\label{fig:m_a}}
\vspace{-0.8em}
\subfigure[]{
\centering
\includegraphics[width=0.235\textwidth]{supervise_label_consistent_e16_pratio0.25.pdf}\label{fig:m_b}}
\vspace{-0.8em}
\subfigure[]{
\centering
\includegraphics[width=0.235\textwidth]{self_badnets_pratio0.05.pdf}\label{fig:m_c}}
\vspace{0.3em}
\subfigure[]{
\centering
\includegraphics[width=0.235\textwidth]{self_label_consistent_e16_pratio0.25.pdf}\label{fig:m_d}}
\vspace{0.3em}
\caption{The t-SNE of poisoned samples in the hidden space generated by different models. \textbf{(a)-(b)}: DNNs trained with supervised learning. \textbf{(c)-(d)}: DNNs trained with self-supervised learning. \textbf{(a)\&(c)}: DNNs under BadNets attack. \textbf{(b)\&(d)}: DNNs under label-consistent attack. No matter under the BadNets or label-consistent attack, poisoned samples tend to cluster together in the hidden space generated by DNNs trained with supervised learning, whereas lie closely to samples with their ground-truth label by those trained with self-supervised learning.}
\label{fig:motivation}
\vspace{-1.5em}
\end{figure}

In this section, we analyze the behavior of poisoned samples from the hidden feature space of attacked models and discuss its inherent mechanism.

\noindent \textbf{Settings.} 
We conduct the BadNets \citep{gu2017badnets} and label-consistent attack \citep{turner2019label} on CIFAR-10 dataset \citep{krizhevsky2009learning} for the discussion. They are representative of poison-label attacks and clean-label attacks, respectively. Specifically, we conduct supervised learning on the poisoned datasets with the standard training process and self-supervised learning on the unlabelled poisoned datasets with SimCLR \citep{chen2020simple}. We visualize poisoned samples in the hidden feature space generated by attacked DNNs based on the t-SNE \citep{van2008visualizing}. More detailed settings are presented in Appendix \ref{sec:motivation_set}.

\noindent \textbf{Results.} 
As shown in Figure \ref{fig:m_a}-\ref{fig:m_b}, poisoned samples (denoted by `black-cross') tend to cluster together to form a separate cluster after the standard supervised training process, no matter under the poison-label attack or clean-label attack. This phenomenon implies why existing poisoning-based backdoor attacks can succeed. Specifically, the excessive learning capability allows DNNs to learn features about the backdoor trigger. Associated with the end-to-end supervised training paradigm, DNNs can shrink the distance between poisoned samples in the feature space and connect the learned trigger-related features with the target label. In contrast, as shown in Figure \ref{fig:m_c}-\ref{fig:m_d}, poisoned samples lie closely to samples with their ground-truth label after the self-supervised training process on the unlabelled poisoned dataset. %Moreover, samples with the same label are still clustered together, although they are not clustered as closely as they were after the supervised learning. 
It indicates that we can prevent the creation of backdoors by self-supervised learning, which will be further introduced in the next section. 

\section{Decoupling-based Backdoor Defense}

\subsection{Preliminaries}
\noindent \textbf{General Pipeline of Backdoor Attacks.} 
Let $\mathcal{D} = \{ (\bm{x}_i, y_i) \}_{i=1}^{N}$ denotes the benign training set, where $\bm{x}_i \in \mathcal{X}= \{0,1,\ldots, 255\}^{C\times W \times H}$ is the image, $y_i \in \mathcal{Y} = \{0,1,\ldots, K\}$ is its label, $K$ is the number of classes, and $y_t \in \mathcal{Y}$ indicates the \emph{target label}. How to generate the poisoned dataset $\mathcal{D}_{p}$ is the cornerstone of backdoor attacks. Specifically, $\mathcal{D}_{p}$ consists of two subsets, including the modified version of a subset of $\mathcal{D}$ and remaining benign samples, $i.e.$, 
$\mathcal{D}_{p} =  \mathcal{D}_{m} \cup \mathcal{D}_{b}$, where $\mathcal{D}_{b} \subset \mathcal{D}$, $\gamma \triangleq \frac{|\mathcal{D}_{m}|}{|\mathcal{D}|}$ is the \emph{poisoning rate}, 
$\mathcal{D}_{m} = \left\{(\bm{x}', y_t)| \bm{x}' = G(\bm{x}), (\bm{x},y) \in \mathcal{D} \backslash \mathcal{D}_{b} \right\}$, and $G: \mathcal{X} \rightarrow \mathcal{X}$ is an attacker-predefined poisoned image generator. For example, $G(\bm{x}) = (\bm{1}-\bm{\lambda}) \otimes \bm{x}+ \bm{\lambda} \otimes \bm{t}$, where $\bm{\lambda} \in [0,1]^{C \times W \times H}$, $\bm{t} \in \mathcal{X}$ is the \emph{trigger} pattern, and $\otimes$ is the element-wise product in the blended attack \citep{chen2017targeted}. Once $\mathcal{D}_{p}$ is generated, it will be sent to users who will train DNNs on it. Hidden backdoors will be created after the training process.

\noindent \textbf{Threat Model.} 
In this paper, we focus on defending against poisoning-based backdoor attacks. The attacker can arbitrarily modify the training set whereas cannot change other training components ($e.g.$, model structure and training loss). For our proposed defense, we assume that defenders can fully control the training process. This is the scenario that users adopt third-party collected samples for training. Note that we do not assume that defenders have a local benign dataset, which is often required in many existing defenses \citep{wang2019neural,zhao2020bridging,li2021neural}. 

\begin{figure*}
    \centering
    \includegraphics[width=0.98\textwidth]{overview.pdf}
    \vspace{-0.5em}
    \caption{The main pipeline of our defense. In the first stage, we train the whole DNN model via self-supervised learning based on label-removed training samples. In the second stage, we freeze the learned feature extractor and adopt all training samples to train the remaining fully connected layers via supervised learning. After that, we filter high-credible samples based on the training loss. In the third stage, we adopt high-credible samples as labeled samples and remove the labels of all low-credible samples to fine-tune the whole model via semi-supervised learning. }
    \vspace{-0.3em}
    \label{fig:overview}
\end{figure*}

\noindent \textbf{Defender's Goals.} 
The defender's goals are to prevent the trained DNN model from predicting poisoned samples as the target label and to preserve the high accuracy on benign samples.

\subsection{Overview of the Defense Pipeline}
In this section, we describe the general pipeline of our defense. As shown in Figure \ref{fig:overview}, it consists of three main stages, including \textbf{(1)} learning a  purified feature extractor via self-supervised learning, \textbf{(2)} filtering high-credible samples via label-noise learning, and \textbf{(3)} semi-supervised fine-tuning. 

Specifically, in the first stage, we remove the label of all training samples to form the unlabelled dataset, based on which to train the feature extractor via self-supervised learning. In the second stage, we freeze the learned feature extractor and adopt all training samples to train the remaining fully connected layers via supervised learning. We then filter $\alpha\%$ high-credible samples based on the training loss. The smaller the loss, the more credible the sample. After the second stage, the training set will be separated into two disjoint parts, including high-credible samples and low-credible samples. We use high-credible samples as labeled samples and remove the label of all low-credible samples to fine-tune the whole model via semi-supervised learning. More detailed information about each stage of our method will be further illustrated in following sections.

\subsection{Learning Purified Feature Extractor via Self-supervised Learning}
\label{sec:purified}

Let $\mathcal{D}_t$ denotes the training set and $f_{\bm{w}}: \mathcal{X} \rightarrow [0,1]^K$ indicates the DNN with parameter $\bm{w}=[\bm{w}_c, \bm{w}_f]$, where $\bm{w}_c$ and $\bm{w}_f$ indicates the parameters of the backbone and the fully connected layer, respectively. In this stage, we optimize $\bm{w}_c$ based on the unlabeled version of $\mathcal{D}_t$ via self-supervised learning, as follows:
\begin{equation}\label{eq:w_c}
    \bm{w}_c^{*} = \arg \min_{\bm{w}_c} \sum_{(\bm{x},y) \in \mathcal{D}_t} \mathcal{L}_1(\bm{x};\bm{w}_c),
\end{equation}
where $\mathcal{L}_1(\cdot)$ indicates the self-supervised loss ($e.g.$, NT-Xent in SimCLR \citep{chen2020simple}). %In this paper, we adopt the one proposed in the SimCLR \cite{chen2020simple}.
Through the self-supervised learning, the learned feature extractor ($i.e.$, backbone) will be purified even if the training set contains poisoned samples, as illustrated in Section \ref{sec:motivation}. 

\subsection{Filtering High-credible Samples via Label-noise Learning}
\label{sec:filtering}
Once $\bm{w}_c^{*}$ is obtained, the user can freeze it and adopt $\mathcal{D}_t$ to further optimize remaining $\bm{w}_f$, $i.e.$,  
\begin{equation}\label{eq:w_f}
    \bm{w}_f^{*} = \arg \min_{\bm{w}_f} \sum_{(\bm{x},y) \in \mathcal{D}_t} \mathcal{L}_2\left(f_{[\bm{w}_c^{*},\bm{w}_f]}(\bm{x}), y \right),
\end{equation}
where $\mathcal{L}_2(\cdot)$ indicates the supervised loss ($e.g.$, cross entropy).

\begin{figure*}[ht]
\centering
\subfigure[Cross-Entropy]{
\centering
\includegraphics[width=0.475\textwidth]{badnets_cifar_resnet_pratio0.2_ce.pdf}}
\hspace{0.3em}
\subfigure[Symmetric Cross-Entropy]{
\centering
\includegraphics[width=0.475\textwidth]{badnets_cifar_resnet_pratio0.2_sce.pdf}}
\vspace{-0.7em}
\caption{Loss values of models under BadNets attack with 20\% poisoning rate trained on CIFAR-10 dataset with the symmetric cross-entropy (SCE) and cross-entropy (CE) in the second stage. All loss values are normalized to $[0, 1]$. As shown in the figure, adopting SCE can significantly increase the loss differences between poisoned samples and benign ones compared with the CE. }
\label{fig:sce}
\end{figure*}

After the decoupling-based training process (\ref{eq:w_c})-(\ref{eq:w_f}), even if the model is (partly) trained on the poisoned dataset, the hidden backdoor cannot be created since the feature extractor is purified. However, this simple strategy suffers from two main problems. Firstly, compared with the one trained via supervised learning, the accuracy of predicting benign samples will have a certain decrease, since the learned feature extractor is frozen in the second stage. %This is also the problem encountered by current self-supervised methods. 
Secondly, poisoned samples will serve as `outliers' to further hinder the learning of the second stage when poison-label attacks appear, since those samples lie close to samples with its ground-truth label instead of the target label in the hidden feature space generated by the learned purified feature extractor. These two problems indicate that \emph{we should remove poisoned samples and retrain or fine-tune the whole model}.

Specifically, we select \emph{high-credible samples} $\mathcal{D}_h$ based on the loss $\mathcal{L}_2(\cdot; [\bm{w}_c^{*},\bm{w}_f^{*}])$. The high-credible samples are defined as the $\alpha\%$ training samples with the smallest loss, where $\alpha \in [0, 100]$ is a hyper-parameter. In particular, we adopt the symmetric cross-entropy (SCE) \citep{wang2019symmetric} as $\mathcal{L}_2(\cdot)$, inspired by the label-noise learning. As shown in Figure \ref{fig:sce}, compared with the CE loss, the SCE can significantly increase the differences between poisoned samples and benign ones, which further reduces the possibility that high-credible dataset $\mathcal{D}_h$ still contains poisoned samples.   

Note that \emph{we do not intend to accurately separate poisoned samples and benign samples}. We only want to ensure that the obtained $\mathcal{D}_h$ contains as few poisoned samples as possible.

\subsection{Semi-supervised Fine-tuning}
\label{sec:semi}
After the second stage, the third-party training set $\mathcal{D}_t$ will be separated into two disjoint parts, including the high-credible dataset $\mathcal{D}_h$ and the low-credible dataset $\mathcal{D}_l \triangleq \mathcal{D}_t \backslash \mathcal{D}_h$. Let $\hat{\mathcal{D}}_l \triangleq \{\bm{x}|(\bm{x}, y) \in \mathcal{D}_l\}$ indicates the unlabeled version of low-credible dataset $\mathcal{D}_l$. We fine-tune the whole trained model $f_{[\bm{w}_c^{*},\bm{w}_f^{*}]}(\cdot)$ with semi-supervised learning as follows:
\begin{equation}
    \min_{\bm{w}} \mathcal{L}_3 (\mathcal{D}_h,\hat{\mathcal{D}}_l;\bm{w} ),
\end{equation}
where %$\bm{w}$ is initialized by the $[\bm{w}_c^{*},\bm{w}_f^{*}]$ and 
$\mathcal{L}_3(\cdot)$ denotes the semi-supervised loss ($e.g.$, the loss in MixMatch \citep{berthelot2019mixmatch}). 

This process can prevent the side-effects of poisoned samples while utilizing their contained useful information, and encourage the compatibility between the feature extractor and the simple classifier via learning them jointly instead of separately. Please refer to Section \ref{sec:ablation_study} for more results. 

\section{Experiments}

\subsection{Experimental Settings}
\noindent \textbf{Datasets and DNNs.}
We evaluate all defenses on two classical benchmark datasets, including CIFAR-10 \citep{krizhevsky2009learning} and (a subset of) ImageNet \citep{deng2009imagenet}. We adopt the ResNet-18 \citep{he2016deep} for these tasks. More detailed settings are presented in Appendix \ref{sec:DD_set}. Besides, we also provide the results on (a subset of) VGGFace2 \citep{cao2018vggface2} in Appendix \ref{sec:results_vggface2}.

\begin{figure*}
    \centering
    \includegraphics[width=0.98\textwidth]{poisoned_samples.pdf}
    \vspace{-0.7em}
    \caption{The illustration of poisoned samples generated by different attacks. }
    \label{fig:poisoned_samples}
\end{figure*}

\noindent \textbf{Attack Baselines.} 
We examine all defense approaches in defending against four representative attacks. Specifically, we select the BadNets \citep{gu2017badnets}, the backdoor attack with blended strategy (dubbed `Blended') \citep{chen2017targeted}, WaNet \citep{nguyen2021wanet}, and label-consistent attack with adversarial perturbations (dubbed `Label-Consistent') \citep{turner2019label} for the evaluation. They are the representative of patch-based visible and invisible poison-label attacks, non-patch-based poison-label attacks, and clean-label attacks, respectively. 

\noindent \textbf{Defense Baselines.} 
We compared our DBD with two defenses having the same defender's capacities, including the DPSGD \citep{du2019robust} and ShrinkPad \citep{li2021backdoor2}. %Specifically, DPSGD adds Gaussian noise with a certain budget during the training process, and ShrinkPad first shrinks the image and then padding to its original size before feeding it into the inference process. 
We also compare with other two approaches with an additional requirement ($i.e.$, having a local benign dataset), including the neural cleanse with unlearning strategy (dubbed `NC') \citep{wang2019neural}, and neural attention distillation (dubbed `NAD') \citep{li2021neural}. %Specifically, NC first synthesizes the potential trigger pattern based on which to conduct the unlearning process, while NAD removes hidden backdoors through the model distillation. 
They are the representative of poison suppression based defenses, preprocessing based defenses, trigger synthesis based defenses, and model reconstruction based defenses, respectively. We also provide results of DNNs trained without any defense (dubbed `No Defense') as another important baseline for reference.

\noindent \textbf{Attack Setups.} 
We use a $2\times 2$ square as the trigger pattern on CIFAR-10 dataset and the $32 \times 32$ Apple logo on ImageNet dataset for the BadNets, as suggested in \citep{gu2017badnets,wang2019neural}. For Blended, we adopt the `Hello Kitty' pattern on CIFAR-10 and the random noise pattern on ImageNet, based on the suggestions in \citep{chen2017targeted}, and set the blended ratio $\lambda = 0.1$ on all datasets. The trigger pattern adopted in label-consistent attack is the same as the one used in BadNets. For WaNet, we adopt its default settings on CIFAR-10 dataset. However, on ImageNet dataset, we use different settings optimized by grid-search since the original ones fail. An example of poisoned samples generated by different attacks is shown in Figure \ref{fig:poisoned_samples}. Besides, we set the poisoning rate $\gamma_1=2.5\%$ for label-consistent attack (25\% of training samples with the target label) and $\gamma_2= 5\%$ for three other attacks. More details are shown in Appendix \ref{sec:attack_set}.

\noindent \textbf{Defense Setups.} 
For our DBD, we adopt SimCLR \citep{chen2020simple} as the self-supervised method and MixMatch \citep{berthelot2019mixmatch} as the semi-supervised method. More details about SimCLR and MixMatch are in Appendix \ref{sec:more_loss}. The filtering rate $\alpha$ is the only key hyper-parameter in DBD, which is set to 50\% in all cases. We set the shrinking rate to 10\% for the ShrinkPad on all datasets, as suggested in \citep{li2021backdoor2,zeng2021deepsweep}. In particular, DPSGD and NAD are sensitive to their hyper-parameters. We \emph{report their best results in each case} based on the grid-search (as shown in Appendix \ref{sec:search_best}). Besides, we split a 5\% random subset of the benign training set as the local benign dataset for NC and NAD. More implementation details are provided in Appendix \ref{sec:defense_set}.

\noindent \textbf{Evaluation Metrics.} 
We adopt the \emph{attack success rate} (ASR) and \emph{benign accuracy} (BA) to measure the effectiveness of all methods\footnote{Among all defense methods, the one with the best performance is indicated in boldface and the value with underline denotes the second-best result.}. Specifically, let $\mathcal{D}_{test}$ indicates the (benign) testing set and $C_{\bm{w}}: \mathcal{X} \rightarrow \mathcal{Y}$ denotes the trained classifier, we have $ASR \triangleq \Pr_{(\bm{x},y) \in \mathcal{D}_{test}}\{C_{\bm{w}}(G(\bm{x}))=y_t|y \neq y_t\}$ and $BA \triangleq \Pr_{(\bm{x},y) \in \mathcal{D}_{test}}\{C_{\bm{w}}(\bm{x})=y\}$, where $y_t$ is the target label and $G(\cdot)$ is the poisoned image generator. In particular, \emph{the lower the ASR and the higher the BA, the better the defense}. 

\subsection{Main Results}
\label{sec:main}

\noindent \textbf{Comparing DBD with Defenses having the Same Requirements.}
As shown in Table \ref{tab:poison}-\ref{tab:clean}, DBD is significantly better than defenses having the same requirements ($i.e.$, DPSGD and ShrinkPad) in defending against all attacks. For example, the benign accuracy of DBD is 20\% over while the attack success rate is 5\% less than that of DPSGD in all cases. Specifically, the attack success rate of models with DBD is less than 2\% in all cases (mostly $< 0.5\%$), which verifies that our method can successfully prevent the creation of hidden backdoors. Moreover, the decreases of benign accuracy are less than $2\%$ when defending against poison-label attacks, compared with models trained without any defense. Our method is even better on relatively larger dataset where all baseline methods become less effective. These results verify the effectiveness of our method.

\begin{table*}[ht]
\centering
\vspace{-2.8em}
\caption{The effectiveness (\%) of defending against three attacks. Note that NC and NAD need an additional local benign dataset, which is not required in DPSGD, ShrinkPad, and our method.}
\scalebox{0.83}{
\begin{tabular}{c|cccccc|cccccc}
\toprule
Dataset $\rightarrow$ & \multicolumn{6}{c|}{CIFAR-10} & \multicolumn{6}{c}{ImageNet} \\ \hline
Attack $\rightarrow$ & \multicolumn{2}{c}{BadNets} & \multicolumn{2}{c}{Blended} &
\multicolumn{2}{c|}{WaNet} & \multicolumn{2}{c}{BadNets} &  \multicolumn{2}{c}{Blended} & \multicolumn{2}{c}{WaNet} \\ \hline
Defense $\downarrow$ & BA & ASR & BA & ASR & BA & ASR & BA & ASR & BA & ASR & BA & ASR \\ \hline
No Defense & 94.92 & 100 & 94.14 & 98.25 & 94.29 & 98.64 & 80.23 & 90.49 & 82.07 & 95.57 & 80.98 & 96.22 \\ \hline
NC & \textbf{94.25} & 2.33 & \textbf{93.01} & 5.70 & \textbf{92.44} & 98.52 & \underline{80.67} & 32.22 & \textbf{82.07} & 27.33 & \underline{80.32} & 99.32 \\
NAD & 90.47 & \underline{1.20} & 89.72 & \textbf{1.51} & \underline{91.83} & \underline{9.49} & 73.49 & \underline{5.81} & 70.88 & \underline{8.15} & 75.27 & 31.43 \\ \hline
DPSGD & 44.24 & 14.56 & 53.05 & 20.68 & 52.13 & 39.78 & 51.43 & 25.40 & 55.12 & 84.17 & 20.72 & 54.12  \\
ShrinkPad & 90.32 & 10.34 & 79.67 & 55.71 & 74.76 & 85.59 & 66.08 & 45.61 & 64.47 & 4.20 & 63.74 & \underline{3.58} \\
DBD (ours) & \underline{92.41} & \textbf{0.96} & \underline{92.18} & \underline{1.73} & 91.20 & \textbf{0.39} & \textbf{80.99} & \textbf{0.26} & \underline{81.63} & \textbf{0.16} & \textbf{82.05} & \textbf{0.33} \\ \bottomrule
\end{tabular}
}
\vspace{-1.5em}
\label{tab:poison}
\end{table*}

\begin{table*}[ht]
\centering
\caption{The effectiveness (\%) of defending against label-consistent attack on CIFAR-10 dataset. %Note that NC and NAD need to have an additional local benign dataset, which is not required in DPSGD, ShrinkPad, and our method.
}
\scalebox{0.9}{
        \begin{tabular}{c|c|c|cc|ccc}
            \toprule
            Attack$\downarrow$, Defense$\rightarrow$                               & Metric$\downarrow$ & Benign & NC               & NAD   & DPSGD & ShrinkPad & DBD (ours)        \\ \hline
            \multirow{2}{*}{\tabincell{c}{Label-Consistent\\($\epsilon=16$)}} & BA                                         & 94.90  & \textbf{94.94}   & 93.04 & 44.28 & \underline{90.67}     & 89.67             \\
                                                              & ASR                                        & 99.33  & \underline{1.31} & 3.95  & 7.97  & 9.60      & \textbf{0.01}     \\ \hline
            \multirow{2}{*}{\tabincell{c}{Label-Consistent\\($\epsilon=32$)}} & BA                                         & 94.82  & \textbf{94.57}   & 90.98 & 69.79 & 90.83     & \underline{91.45} \\
                                                              & ASR                                        & 99.19  & \underline{1.36} & 6.54  & 9.06  & 13.04     & \textbf{0.34}     \\ \bottomrule
        \end{tabular}
    }
\label{tab:clean}
\vspace{-0.9em}
\end{table*}

\noindent \textbf{Comparing DBD with Defenses having Extra Requirements.}
We also compare our defense with two other methods ($i.e.$, NC and NAD), which have an additional requirement that defenders have a benign local dataset. As shown in Table \ref{tab:poison}-\ref{tab:clean}, NC and NAD are better than DPSGD and ShrinkPad, as we expected, since they adopt additional information from the benign local dataset. In particular, although NAD and NC use additional information, our method is still better than them, even when their performances are tuned to the best while our method only uses the default settings. Specifically, the BA of NC is on par with that of our method. However, it is with the sacrifice of ASR. Especially on ImageNet dataset, NC has limited effects in reducing ASR. In contrast, our method reaches the smallest ASR while its BA is either the highest or the second-highest in almost all cases. These results verify the effectiveness of our method again.

\begin{table*}[!t]
   \centering
   \vspace{-0.5em}
   \caption{The ablation study of our proposed method.}
   \scalebox{0.95}{
      \begin{tabular}{c|cccc|cc|cc}
         \toprule
         Attack $\rightarrow$                       & \multicolumn{2}{c}{BadNets} & \multicolumn{2}{c|}{Blended} & \multicolumn{2}{c|}{Label-Consistent} & \multicolumn{2}{c}{WaNet}\\ \hline
         Defense $\downarrow$, Metric $\rightarrow$ & BA                          & ASR                          & BA                                    & ASR                                  & BA                                    & ASR   & BA    & ASR\\ \hline
         No Defense                                 & 94.92                       & 100                          & 94.14                                 & 98.25                                & 94.82                                 & 99.19 & 94.29 & 98.64\\ \hline
         DBD without SS   & \textbf{93.66}                       & 100                          & \textbf{93.47}                                 & 99.93                                & 90.70                                 & 98.50  & 81.91 & 98.40\\ \hline
         SS with CE                                 & 82.25                       & 5.20                         & 81.85                                 & 12.19                                & 82.08                                 & 5.87  & 80.29 & 9.48\\
         SS with SCE                                & 82.34                       & 5.12                         & 82.30                                 & 6.24                                 & 81.81                                 & 5.43  & 81.15 & 7.08\\ \hline
         SS with SCE + Tuning                       & 78.94                       & \underline{4.02}                         & 78.78                                 & \underline{3.70}                                 & 77.44                                 & \underline{1.44}  & 78.51 & \underline{5.87}\\ \hline
         DBD (ours)                                 & \underline{92.41}                       & \textbf{0.96}                         & \underline{92.18}                                 & \textbf{1.73}                                 & \textbf{91.45}                                 & \textbf{0.34}  & \textbf{91.20} & \textbf{0.39}
         \\ \bottomrule
      \end{tabular}
   }
   \begin{tablenotes}
      \footnotesize
      \item[1] $^{1}$DBD without SS: use standard supervised training to obtain the feature extractor adopted in DBD.
      \item[2] $^{2}$SS with CE: freeze the learned backbone and train remaining FC layers on all samples with CE loss.
      \item[3] $^{3}$SS with SCE: freeze the learned backbone and train remaining FC layers on all samples with SCE loss.
      \item[4] $^{4}$SS with SCE + Tuning: fine-tune FC layers of the model in `SS with SCE' on high-credible samples.
   \end{tablenotes}
   \label{tab:ablation}
   \vspace{-1em}
\end{table*}

\noindent \textbf{Results. }
As shown in Figure \ref{fig:gamma}, our method can still prevent the creation of hidden backdoors even when the poisoning rate reaches 20\%. Besides, DBD also maintains high benign accuracy. In other words, our method is effective in defending attacks with different strengths. %Those results verify the effectiveness of our method again. 

\subsection{Ablation Study}
\label{sec:ablation_study}

There are four key strategies in DBD, including \textbf{(1)} obtaining purified feature extractor, \textbf{(2)} using SCE instead of CE in the second stage, \textbf{(3)} reducing side-effects of low-credible samples, and \textbf{(4)} fine-tuning the whole model via semi-supervised learning. Here we verify their effectiveness.

\noindent \textbf{Settings. }
We compare the proposed DBD with its four variants, including \textbf{(1)} DBD without SS, \textbf{(2)} SS with CE, \textbf{(3)} SS with SCE, and \textbf{(4)} SS with SCE + Tuning, on the CIFAR-10 dataset. Specifically, in the first variant, we replace the backbone generated by self-supervised learning with the one trained in a supervised fashion and keep other parts unchanged. In the second variant, we freeze the backbone learned via self-supervised learning and train the remaining fully-connected layers with cross-entropy loss on all training samples. The third variant is similar to the second one. The only difference is that it uses symmetric cross-entropy instead of cross-entropy to train fully-connected layers. The last variant is an advanced version of the third one, which further fine-tunes fully-connected layers on high-credible samples filtered by the third variant. %Unless otherwise specified, all settings used in this part are the same as those used in Section \ref{sec:main}.

\noindent \textbf{Results. }
As shown in Table \ref{tab:ablation}, we can conclude that decoupling the original end-to-end supervised training process is effective in preventing the creation of hidden backdoors, by comparing our DBD with its first variant and the model trained without any defense. Besides, we can also verify the effectiveness of SCE loss on defending against poison-label backdoor attacks by comparing the second and third DBD variants. Moreover, the fourth DBD variant has relatively lower ASR and BA, compared with the third one. This phenomenon is due to the removal of low-credible samples. It indicates that reducing side-effects of low-credible samples while adopting their useful information is important for the defense. We can also verify that fine-tuning the whole model via semi-supervised learning is also useful by comparing the fourth variant and the proposed DBD.

\subsection{Resistance to Potential Adaptive Attacks}
\label{sec:adaptive}

In our paper, we adopted the classical defense setting that attackers have no information about the defense. Attackers may design adaptive attacks if they know the existence of our DBD. The most straightforward idea is to \emph{manipulate the self-supervised training process} so that poisoned samples are still in a new cluster after the self-supervised learning. However, \emph{attackers are not allowed to do it} based on our threat model about adopting third-party datasets. Despite this, attackers may design adaptive attacks by optimizing the trigger pattern to make poisoned samples still in a new cluster after the self-supervised learning if they can know the model structure used by defenders, as follows:

\noindent \textbf{Problem Formulation.} For a $K$-classification problem, let $\mathcal{X}'=\{\bm{x}_i\}_{i=1}^M$ indicates the benign images selected for poisoning,  $\mathcal{X}_{j}=\{\bm{x}_i\}_{i=1}^{N_j}$ denotes the benign images with ground-truth label $j$, and $g$ is a trained backbone. Given an attacker-predefined poisoned image generator $G$, the \emph{adaptive attack} aims to optimize a trigger pattern $\bm{t}$ by minimizing the distance between poisoned images while maximizing the distance between the center of poisoned images and centers of clusters of benign images with different label, $i.e.,$
\begin{equation}
\min_{\bm{t}} \frac{1}{M}\sum_{\bm{x} \in \mathcal{X}'} d\left(g(G(\bm{x};\bm{t})), \overline{g}')\right) - \frac{1}{K}\sum_{i=1}^K d\left(\overline{g}', \overline{g_{i}}\right),     
\end{equation}
    
where $\overline{g}' \triangleq \frac{1}{M}\sum_{\bm{x} \in \mathcal{X}'} g(G(\bm{x};\bm{t}))$, $\overline{g_i} \triangleq \frac{1}{N_i} \sum_{\bm{x} \in \mathcal{X}_i} g(\bm{x})$, and $d$ is a distance metric. 

\noindent \textbf{Settings.}
We adopt the CIFAR-10 dataset and use the $\ell^2$ norm as the distance metric to conduct the experiment. Specifically, we assume that attackers have the entire benign dataset, based on which they can train a backbone adopted in the first stage of our DBD. We use the Adam optimizer to solve the above optimization problem for 100 epochs with a learning rate of 0.1. The trigger size is set to $32\times32$, which means the attacker can completely modify the content of poisoned samples, regardless of its original semantic information and the stealthiness of the attack. This setting is to ensure the attack ability, since clustering poisoned samples together is very difficult in self-supervised learning.

\noindent \textbf{Results.}
The adaptive attack works well when there is no defense (BA=94.96\%, ASR=99.70\%). However, this attack still fails to attack our DBD (BA=93.21\%, ASR=1.02\%). In other words, our defense is resistant to this adaptive attack. It is most probably because the trigger optimized based on the backbone is far less effective when the model is retrained since model parameters are changed due to the random initialization and the update of model weights during the training process.

\comment{
\subsection{A Brief Description of Additional Experiments in the Appendix}
Except for above experiments, we also provide additional results of defending against attacks with different hyper-parameters in Appendix \ref{sec:smaller_p}-\ref{sec:diff_trigger} and the resistance to adaptive attacks in Appendix \ref{sec:adaptive}. %More details about training facilities and codes are also included in Appendix \ref{sec:train_f}-\ref{sec:codes}.
} 

\vspace{-0.3em}
\section{Conclusion}
\vspace{-0.2em}

The mechanism of poisoning-based backdoor attacks is to establish a latent connection between trigger patterns and the target label during the training process. In this paper, we revealed that this connection is learned mostly due to the end-to-end supervised training paradigm. Motivated by this understanding, we proposed a decoupling-based backdoor defense, which first learns the backbone via self-supervised learning and then the remaining fully-connected layers by the classical supervised learning. We also introduced the label-noise learning method to determine high-credible and low-credible samples, based on which we fine-tuned the whole model via semi-supervised learning. Extensive experiments verify that our defense is effective on reducing backdoor threats while preserving high accuracy on predicting benign samples. %Our paper may provide a new angle about understanding the learning dynamic of poisoning-based backdoor attacks.

\newpage

\section*{Acknowledgments}
Baoyuan Wu is supported in part by the National Natural Science Foundation of China under Grant 62076213, the University Development Fund of the Chinese University of Hong Kong, Shenzhen under Grant 01001810, and the Special Project Fund of Shenzhen Research Institute of Big Data under Grant T00120210003. 
Zhan Qin is supported in part by the National Natural Science Foundation of China under Grant U20A20178, the National Key Research and Development Program of China under Grant 2020AAA0107705, and the Research Laboratory for Data Security and Privacy, Zhejiang University-Ant Financial Fintech Center.
Kui Ren is supported by the National Key Research and Development Program of China under Grant 2020AAA0107705.

\section*{Ethics Statement}
DNNs are widely adopted in many mission-critical areas ($e.g.$, face recognition) and therefore their security is of great significance. The vulnerability of DNNs to backdoor attacks raises serious concerns about using third-party training resources. In this paper, we propose a general training pipeline to obtain backdoor-free DNNs, even if the training dataset contains poisoned samples. This work has no ethical issues in general since our method is purely defensive and does not reveal any new vulnerabilities of DNNs. However, we need to mention that our defense can be adopted only when training with untrusted samples, and backdoor attacks could happen in other scenarios. People should not be too optimistic about eliminating backdoor threats. 

\section*{Reproducibility Statement}
The detailed descriptions of datasets, models, and training settings are in Appendix \ref{sec:motivation_set}-\ref{sec:search_best}. We also describe the computational facilities and cost in Appendix \ref{sec:train_f}-\ref{sec: computational cost}. Codes of our DBD are also open-sourced.

\newpage

\end{document}