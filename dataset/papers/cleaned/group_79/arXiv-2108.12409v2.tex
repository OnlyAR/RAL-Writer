\title{Train Short, Test Long: Attention with Linear Biases Enables Input Length Extrapolation}

\begin{document}

\maketitle

\begin{abstract}

Since the introduction of the transformer model by \citet{vaswani}, a fundamental question has yet to be answered:  how does a model achieve extrapolation at inference time for sequences that are longer than it saw during training? 
We first show that extrapolation can be enabled by simply changing the position representation method, though we find that current methods do not allow for \emph{efficient} extrapolation. 
We therefore introduce a simpler and more efficient position method, Attention with Linear Biases (ALiBi). ALiBi does not add positional embeddings to word embeddings;  instead, it biases query-key attention scores with a penalty that is proportional to their distance. We show that this method trains a 1.3 billion parameter model on input sequences of length 1024 that extrapolates to input sequences of length 2048, achieving the same perplexity as a sinusoidal position embedding model trained on inputs of length 2048 but training 11\% faster and using 11\% less memory. 
ALiBi's inductive bias towards recency also leads it to outperform multiple strong position methods on the WikiText-103 benchmark.%
\footnote{Code \& models: \url{https://github.com/ofirpress/attention_with_linear_biases}}

\end{abstract}

\section{Introduction}
\label{sec:intro}

When constructing a transformer-based language model, a major design decision is the length of training sequences, denoted $L$ herein, which has to date  been equivalent to the length of inference sequences.  More context, achieved by larger $L$, improves predictions at inference time.  But longer sequences are more expensive to train on.\footnote{Figure~\ref{fig:wt103_train_speeds_all} in the appendix plots training speed, in words per second, against $L$.}  

Before transformers, RNN language models were trained on shorter-$L$ sequences and assumed to generalize to longer contexts at inference time~\citep{Mikolov2010RecurrentNN, Mikolov2012ContextDR, zaremba2014recurrent}.  \citet{vaswani}, introducing the transformer, speculated that it ``may [...] extrapolate to sequence lengths longer than the ones encountered during training.''  
We define \emph{extrapolation} as a model's ability to continue performing well as the number of input tokens during validation increases beyond the number of tokens on which the the model was trained. 
We find that transformer language models (LMs)  that use sinusoidal position embeddings have very weak  %
extrapolation abilities; see Figure~\ref{fig:wt103_extra}.

\begin{figure}[h]
\centering
\begin{subfigure}{.5\textwidth}
  \centering
  \includegraphics[width=\linewidth]{figures/wt103_extra_512.pdf}
\end{subfigure}%
\begin{subfigure}{.5\textwidth}
  \centering
  \includegraphics[width=\linewidth]{figures/wt103_extra_1024.pdf}
\end{subfigure}
\caption{Extrapolation:
as the (validation-set's) input sequence gets longer ($x$-axis), current position methods (sinusoidal, rotary, and T5) show degraded perplexity ($y$-axis, lower is better), but our method (\S\ref{sec:ourmethod}) does not.  Models were trained on WikiText-103 with sequences of $L=$ 512 (left) or $L=$ 1,024 (right) tokens.  T5 ran out of memory on our 32GB GPU. For more detail on exact perplexities and runtimes, see Tables~\ref{tab:appendix:wt103_extra_512} and~\ref{tab:appendix:wt103_extra_1024} in the appendix.}

\label{fig:wt103_extra}
\end{figure}

We demonstrate that this failure to extrapolate is caused by the position embedding method.  As shown in Figure~\ref{fig:wt103_extra}, recent alternatives to the original sinusoidal position method \citep{roformer,t5} have improved extrapolation.  However, the better of these, the T5 bias, is considerably slower than the sinusoidal approach and uses extra memory and parameters (Figure~\ref{fig:wt103_speed_mem}).

We therefore introduce Attention with Linear Biases (ALiBi) to facilitate  efficient extrapolation. %
ALiBi negatively biases attention scores with a linearly decreasing penalty proportional to the distance between the relevant key and query. Our simple approach eliminates position embeddings. %

Compared to a sinusoidal model trained on the same input length, our method requires no additional runtime or parameters and incurs a negligible (0--0.7\%) memory increase.  ALiBi can be implemented by changing only a few lines of existing transformer code. 

Using ALiBi, a transformer LM can be trained on short-$L$ sequences and therefore at much lower cost, and it can still be reliably applied to long sequences at runtime. For example, a 1.3 billion parameter LM trained on $L=$ 1024 tokens with ALiBi achieves the same perplexity as a sinusoidal model trained on $L=$ 2048 when both are tested on sequences of 2048 tokens, even though \textit{our model is 11\% faster and uses 11\% less memory. }%

Though performance peaks at around two times the number of tokens that the model was trained on, ALiBi maintains strong performance even on sequences of length 10,000. 
In recently explored settings where NLP training examples are given as context to an LM \citep{gpt3}, our approach will allow exposure to more examples. Additionally, it enables generation of longer outputs.

\section{Current Approaches Do Not Extrapolate Efficiently }
\label{sec:act2}

 

We show for the first time that the sinusoidal position method, which technically should be able to extrapolate, in practice has very limited extrapolation capabilities. Though the rotary position method improves over the sinusoidal one, it still does not achieve satisfying results.  Holding everything else constant, we are the first to observe that the T5 bias method leads to better extrapolation than either of these, and so we conclude that extrapolation ability depends heavily on the position embedding.  Unfortunately, the T5 bias is computationally costly (Figure~\ref{fig:wt103_speed_mem}).

\subsection{Background and Experimental Setup}

A transformer LM receives a list of tokens 
and outputs a probability distribution representing its prediction for the next token.  We call the input list the \textit{current input subsequence} since the inputs to language models are typically subsequences from (much longer) training or evaluation sequences.  During both training and perplexity evaluation (i.e., scoring a fixed sequence), many predictions can be calculated at once; this is done using a ``causal mask'' that ensures each position's prediction is influenced only by tokens to its left.  Let $L$ be the length of each input subsequence during training; it includes $L$ predictions, which on average have access to $\frac{L+1}{2}$ tokens of (left) context.  %
To explore a model's extrapolation abilities, we are interested in cases where sequences of length $L_{\textit{valid}} > L$ are considered at evaluation time. %
When $L$ differs between inference and training, we use \lt to refer to the length of subsequences during training and \li  to refer to their length at validation. 

\paragraph{Nonoverlapping Inference}
To train on or evaluate a sequence longer than $L$ tokens, it is typical to segment the sequence into $L$-length subsequences and train on or evaluate them independently.  Unless otherwise stated, we use nonoverlapping inference to report perplexity scores. %

\paragraph{Extrapolation During Inference} 
Formally, the functions that define a transformer layer are agnostic to input length;\footnote{These include the embedding lookup, feedforward sublayer, and softmax layer, which act independently on vector inputs, as well as the attention sublayers, whose parameters do not depend on input length (and which must handle variable-length inputs, e.g., due to causal masking).  %
} they map from some arbitrary, unfixed number of input vectors to the same number of output vectors.  When transformers are applied to data that is inherently sequential, like text, %
positional information is injected into the inputs in various ways. 

\citet{vaswani}
discussed two options for embedding positions into vectors to be added to word %
embeddings:  learning embeddings for specific positions and unlearned sinusoidal embeddings.  They observed similar performance between these two but preferred the sinusoidal approach, which they argued might extrapolate to longer input sequences during inference.  %
We find that this model cannot extrapolate to more than a few dozen tokens beyond $L$.\footnote{The learned positional embedding approach does not have a way to encode positions greater than $L$; it therefore has no ability to extrapolate.}

\paragraph{Experiment Setup}
We first test the extrapolation abilities of various position methods on the WikiText-103 corpus~\citep{pointer} using the transformer language model of~\cite{baevski}. 
We use this model because of its prominent role in recent  language modeling developments~\citep{khandelwal20generalization, shortformer}. The training set is about 103 million tokens from  English Wikipedia (half a gigabyte).  The  model has $16$ transformer layers of dimension $1024$, with $8$ heads, and a feedforward inner dimension of $4096$. This model ties the word embedding and softmax matrices~\citep{tying, inan2017}. %
In our experiments, other than varying the position method and training subsequence length, we modify no other hyperparameters, including the random seed and number of training epochs (205). 

\begin{figure}
\centering
\begin{subfigure}{.28\textwidth}
  \centering
  \includegraphics[width=\linewidth]{figures/wt103_Training-Speed.pdf}
\end{subfigure}%
\begin{subfigure}{.28\textwidth}
  \centering
  \includegraphics[width=\linewidth]{figures/wt103_Inference-Speed.pdf}
\end{subfigure}
\begin{subfigure}{.28\textwidth}
  \centering
  \includegraphics[width=\linewidth]{figures/wt103_Training-Memory-Usage.pdf}
\end{subfigure}
\begin{subfigure}[]{.14\textwidth}
\centering
\raisebox{3mm}{\includegraphics[width=\linewidth]{figures/legend.pdf}}
\end{subfigure}
\caption{A comparison of batched training, inference speed and memory use of the sinusoidal, rotary, T5 bias, and our ALiBi position methods. The speed differences between our method and the sinusoidal are within 1\% during training and 3\% for inference, which is insignificant on our hardware. ALiBi uses 100MB of extra memory when training on input lengths 1024 and 3072 in this setting. Memory usage is lower in all approaches when training on 3072 tokens (compared to 1024) since we break batches into multiple updates. See Table~\ref{tab:baselines_speed_mem} in the appendix for exact numbers. 
}
\label{fig:wt103_speed_mem}
\end{figure}

\subsection{Measuring Extrapolation}
\paragraph{Sinusoidal Position Embeddings}
Sinusoidal position embeddings (\citealp{vaswani}; \S 3.5) are constant, non-learned vectors that are added to token embeddings on input to the first layer of the transformer. They are frequently used in transformer language modeling~\citep{baevski,lewis2021base} and machine translation~\citep{vaswani,ott2018scaling} models. 
We first consider the unmodified model of~\cite{baevski}, which uses sinusoidal position embeddings, and train it on $L=512$ tokens; we then run inference with it on the validation set on $L+k$ tokens, with $k$ ranging from 0 to 15,000.
Figure~\ref{fig:wt103_extra} (left) and the corresponding Table~\ref{tab:appendix:wt103_extra_512} (in the appendix) show that while the model improves  perplexity up to $k=20$, performance stops improving and stays steady from $k=20$ to $k=50$ and then begins degrading. 
Similar results are obtained for a model trained with $L=1024$ tokens (Figure~\ref{fig:wt103_extra} (right) and Table~\ref{tab:appendix:wt103_extra_1024} in the appendix). That model improves for up to $\li = \lt + 50$ tokens, after which performance declines. 

\paragraph{Rotary Position Embeddings}
The rotary method was introduced by~\cite{roformer} and has recently been popularized by the open source GPT-3~\citep{gpt3} implementation GPT-J~\citep{gpt-j}. Instead of adding sinusoidal embeddings at the bottom of the transformer, they multiply the keys and queries of every attention layer by sinusoidal embeddings. 

Unlike the sinusoidal or learned positional embedding approach, the rotary method injects position information into the model at every layer, not just at the initial one.  In addition, it adds no position information to the values of the self-attention sublayer. The output of a self-attention sublayer is a linearly transformed, weighted sum of the input value vectors; therefore, by not inserting position information into the values, the outputs of each transformer-layer contain no explicit position information. 
We suspect that this segregation of position information may be beneficial for extrapolation, and we draw inspiration from it in the design of our method (\S\ref{sec:ourmethod}).

We apply the rotary position embedding method to our Baevski \& Auli baseline.\footnote{Our rotary method implementation is based on the code in \url{https://github.com/JunnYu/RoFormer_pytorch}, which is linked to from the official repository of~\cite{roformer}: (\url{https://github.com/ZhuiyiTechnology/roformer}). After we finished running our experiments with the rotary method, we were informed that the runtime of the code linked above could be optimized, making it only 2\% slower than the sinusoidal approach. This optimization would not change extrapolation performance.}
The perplexity results (Figure~\ref{fig:wt103_extra} and Appendix Tables~\ref{tab:appendix:wt103_extra_512} and~\ref{tab:appendix:wt103_extra_1024}) are better than the sinusoidal approach: the model with $L=512$ ($L=1024$) improves perplexity with up to $k=200$ ($k=100$) more tokens than it saw during training, but this comes at the cost of slower training and inference (Figure~\ref{fig:wt103_speed_mem}). %

\paragraph{T5 Bias}
Though most models use trained or sinusoidal position embeddings, the T5 model of~\cite{t5} uses a relative position method~\citep{shaw,Huang2019MusicTG} that adds no  position information to word embeddings (as in the previous method). Instead, it  %
modifies the way attention values are computed. We refer to this as the ``T5 bias'' method.\footnote{This method is similar to the one used in~\citet[Equation 7]{parikh-etal-2016-decomposable}.} To compute attention values in the unmodified transformer, we compute the dot product of every query with every relevant key %
and then softmax these attention values. In this method, we compute the attention values as before, %
but then we add a learned, shared bias to each query-key score that is dependent on just the distance between the query and key. Therefore, all query-key scores where the query and key distance are zero (i.e., the query and key represent the same token) get a specific learned bias, all scores where the query and key are one word away get a different learned bias, and so on, up to a certain point, from where multiple different distances share the same learned bias (which might be beneficial for extrapolation). %
As in the rotary method, the T5 bias injects position information into the model at every layer and integrates no explicit position information into the %
self-attention value vectors. 

\cite{t5} propose that the T5 bias may allow extrapolation, but they did not report experiments testing this.  Here, we show that the T5 bias does allow language models to extrapolate.
We do this by again modifying the Baevski \& Auli model, this time to insert the T5 bias into it.\footnote{Our T5 bias implementation is based on the one used in HuggingFace Transformers~\citep{huggingface}, which in turn is based on the official Mesh Tensorflow T5 code. }

As Figure~\ref{fig:wt103_extra} shows, the T5 bias improves perplexity with longer sequences than the ones it was trained on, i.e.,  $k=600$ ($k=800$) extra tokens for a model trained on $L=512$ ($L=1024$) input tokens.  Unfortunately, this impressive performance comes at a cost: training is at least twice as slow as with the sinusoidal model. Therefore, this model's extrapolation ability provides no efficiency advantage. For example, to do inference on 1024 tokens, we could either train the sinusoidal model with \lt = 1024 or train the T5 bias model on \lt = 512 tokens and extrapolate to 1024 for inference. However, the \lt = 1024 sinusoidal model runs at 28.5k words per second (WPS), while the \lt = 512 T5 bias model runs at 14.4k WPS (Appendix Table~\ref{tab:baselines_speed_mem}), so there is no speedup when training on shorter sequences with this method.\footnote{ \citet{narang2021transformer} benchmarked the T5 bias as being just 8.7\% slower than the sinusoidal approach; thus,   while always incurring a runtime penalty, this method's runtime could be faster depending on the choice of hardware and software frameworks used. Narang et al. used the Tensorflow T5 library running on TPUs, while we used the PyTorch Fairseq library running on GPUs. }

\section{Attention with Linear Biases (ALiBi)}
\label{sec:ourmethod}

\begin{figure}
\begin{center}
\includegraphics[width=.48\textwidth]{figures/fig1.pdf} %
\end{center}
\caption{When computing attention scores for each head, our linearly biased attention method, ALiBi,  adds a constant bias (right) to each attention score ($\mathbf{q}_i \cdot \mathbf{k}_j$, left). As in the unmodified attention sublayer, the softmax function is then applied to these scores, and the rest of the computation is unmodified. \textbf{$\textbf{m}$ is a head-specific scalar} that is set and not learned throughout training. We show that our method for setting $m$ values generalizes to multiple text domains, models and training compute budgets. %
When using ALiBi, we do \emph{not} add positional embeddings at the bottom of the network.  }
\label{fig:1}
\end{figure}

In the transformer model of \citet{vaswani}, position embeddings are added to the word embeddings at the bottom of the network. %
For an input subsequence of length $L$, the attention sublayer %
computes the attention scores for the $i$th
query $\mathbf{q}_i \in \mathbb{R}^{1\times d}$, ($1 \leq i \leq L$) in each head, given the first $i$ keys $\mathbf{K} \in \mathbb{R}^{i\times d}$, where $d$ is the head dimension:%
\begin{equation*}
\text{softmax}(\mathbf{q}_i \mathbf{K}^\top)
\end{equation*}
These attention scores are then multiplied by the values to return the output of the attention sublayer.\footnote{For simplicity we omit the key, query, value and final output projections, dropout, and the scaling factor.}

When using ALiBi, we do not add position embeddings at any point in the network. %
The only modification we apply is after the query-key dot product, where we add a static, non-learned bias:\footnote{The ALiBi bias is not multiplied by the $\sqrt{d_k}$ scaling factor from Equation 1 of~\citet{vaswani}.}
\begin{equation*}
\text{softmax}(\mathbf{q}_i \mathbf{K}^\top + m \cdot [-(i-1), ..., -2, -1, 0]), 
\end{equation*}
where scalar $m$ is a head-specific slope fixed before training.
Figure~\ref{fig:1} offers a visualization. 

For our models with 8 heads, the slopes that we used are the geometric sequence: %
${\frac{1}{2^1}, \frac{1}{2^2}, ..., \frac{1}{2^8}}$. 
For models that require 16 heads, we interpolate those 8 slopes by geometrically averaging every consecutive pair, resulting in the geometric sequence that starts at $\frac{1}{\sqrt{2}}$ and has the ratio of $\frac{1}{\sqrt{2}}$: ${\frac{1}{2^{0.5}}, \frac{1}{2^1}, \frac{1}{2^{1.5}}, ..., \frac{1}{2^{8}}}$. In general, for $n$ heads, our set of slopes is the geometric sequence that starts at %
$2^{\frac{-8}{n}}$
and uses that same value as its ratio.  

In \S\ref{sec:results}, we observe that this set of slopes works on a wide variety of text domains and model sizes. Therefore, we do not believe that it is necessary to tune these slope values every time a new model is trained on a new dataset. This makes our method similar to the sinusoidal approach, where the hyperparameters (the start and end of the geometric progression of wavelengths) were set once by~\citet{vaswani} and then reused in different models of different sizes on different datasets.  

\al has an inductive bias towards recency; it penalizes attention scores between distant query-key pairs, with the penalty increasing as the distance between a key and a query grows. The different heads increase their penalties at different rates, depending on the slope magnitude. 

We initially experimented with making the slopes trainable, but this did not yield strong extrapolation results.\footnote{In our experiments, trainable slopes also slowed down the training speed by 3\%.} %
A brief manual exploration of around ten slope sets led us to discover the set of slopes that we finally picked. Our main insight from this exploration is that the slope sets that work best are those with slopes in the $(0,1)$ range, with the slopes' density increasing as we get closer to $0$. 
We also found our method to be robust to slope choice. Even randomly sampling from the exponential distribution worked well in some cases (although that method had high variance). %

Since ALiBi is a relative position method, we add position information at every layer to the keys and queries but not to the values, as is done in the T5 bias and rotary methods. We hypothesize that these properties might be beneficial for extrapolation. %

\paragraph{Implementation.}
ALiBi is easy to implement, with all changes accomplished in a few lines of code. We implement it by modifying the mask matrix by adding the linear biases to it (in practice, when training a transformer LM, query $\mathbf{q}_i$ attends only to keys $1$ to $i$; this is implemented by adding a mask matrix to the query-key dot product before the softmax operation is applied). This means that there is no runtime penalty when using our method since we add no operations to the network.%

Compared to the sinusoidal model trained on the same input lengths, AliBi incurs a memory increase (up to 100MB in some of our experiments): in the unmodified transformer, the mask is of size $L\times L$; when using ALiBi, the mask is a slightly larger $n \times L\times L$ (where $n$ is the number of heads) since the linear biases added for each head uses a different slope. But, as we show, ALiBi enables training on much smaller sequences while still achieving (and occasionally surpassing) results obtained using sinusoidal embeddings on longer sequences, which saves multiple gigabytes of memory. 
\section{Results}
\label{sec:results}
We first show that on WikiText103 ALiBi is efficient and enables training models with short input subsequences that outperform strong baselines even when the ALiBi models extrapolate to more than six times the number of tokens that they were trained on. 
We then take the same hyperparameters for our method (the set of slopes) that worked on WikiText-103 and show that -- with no modification -- they provide strong results on a dataset in a very different domain: books. 
Finally, we show that a 1.3B parameter model trained with AliBi on a much larger (461 GB) dataset with much more compute provides a superior alternative to the sinusoidal method since it achieves similar perplexity scores while running faster and using less memory (since it is trained on shorter inputs). 

While multiple alternatives to the position methods presented in~\cite{vaswani} have been proposed, few have been adopted in large (1B or more parameter) LMs since  that setting is much more challenging than the smaller scale experiments. GPT-3 and Jurassic-1~\citep{J1WhitePaper} use the learned position embedding method from Vaswani et al., and GPT-J uses the rotary method.
Our results on the 1.3B parameter model show our method's ability to generalize to larger models, dataset sizes and training durations without retuning the hyperparameter. 

\subsection{Results on WikiText-103 and Toronto BookCorpus}

\begin{figure}[h]
\begin{center}
\includegraphics[width=.65\textwidth]{figures/wt103_extra_lsb_all.pdf} %
\end{center}
\caption{\al models trained and evaluated on varying sequence lengths on the WikiText-103 validation set and the sinusoidal baseline (not evaluated on longer sequences). All of our models outperform the sinusoidal ones even when trained on fewer tokens. Appendix Table~\ref{tab:LSB_wt103} has exact perplexities, more \al models (trained on fewer tokens), and results for rotary and T5 bias models. }
\label{fig:wt103_extra_lsb_all}
\end{figure}

We first develop our method on the WikiText-103 corpus~\citep{pointer}, replacing the sinusoidal position embeddings in the language model of~\cite{baevski} with \al. 

Figure~\ref{fig:wt103_extra_lsb_all} (and the corresponding Appendix Table~\ref{tab:LSB_wt103}) show our results for models trained with varying numbers of input subsequence tokens ($L$), extrapolating to longer subsequence lengths on the validation dataset. 
Our first observation is that, without extrapolation, for every $L$, our models outperform  those using the sinusoidal method, sometimes by a  significant %
amount. For example, the Baevski \& Auli model achieves 18.67$\pm$0.24 (std.~dev.) perplexity when trained with $L = 3072$ input tokens, but our $L=3072$ model achieves 17.60 perplexity (when both models evaluate with \li = 3072).

Our second observation is that all of our models can extrapolate, and they obtain improved perplexity scores when handling more tokens than they observed during training. For example, our model trained on 512 tokens (which achieves 19.73 perplexity when evaluating subsequences of length 512 in the development set) achieves a perplexity score of 18.40 on the development set when extrapolating to subsequences of length 3072. Surprisingly, this surpasses the score that the $L=3072$ sinusoidal model obtains on the development set by a statistically significant margin. 
Note that all our models trained on $L=512$ to $L=2048$ outperform the sinusoidal baseline trained on $L=3072$ when extrapolating to \li = 3072 even though those models all take much less time to train since they train on shorter subsequences (Appendix Figure~\ref{fig:wt103_train_speed_vs_ppl} compares training speed to perplexity for these models)! The $L=512$ model is 1.84 times faster to train and yet still outperforms the $L=3072$ sinusoidal model when extrapolating to $\li = 3072$. In addition, training the $L=3072$ sinusoidal model requires a GPU with more than 16 GB of memory to fit the large attention matrices, which our $L=512$ outperforms even though it can be trained on a GPU with much less memory due to much smaller attention matrices. 

Additionally,  Table~\ref{tab:LSB_wt103} (in the appendix) also shows that, for $L$s  of 1024 and 3072, our method performs better than the rotary and T5 bias models even when \li = $L$ (i.e., no extrapolation is occurring).
Figure~\ref{fig:wt103_extra} (and the corresponding Appendix Tables~\ref{tab:appendix:wt103_extra_512} and~\ref{tab:appendix:wt103_extra_1024}) more broadly explore our method vs.~the other position methods. They show that the T5 bias (the best of the baselines) improves perplexity until \li is around $2L$, but on the WikiText-103 dataset our method continually improves perplexity until at least around $3L$, with the $L=512$ model improving perplexity even when \li exceeds 12k tokens. Even when unable to improve perplexity given longer sequences, \al always maintains strong performance as more tokens are added. 

Appendix Table~\ref{tab:wt103_test} shows that our results on the validation set also transfer to the test set of WikiText-103.
Currently, almost all models that present results on WikiText-103 use sliding window evaluation  (defined in \S\ref{sec:analysis}) to compute perplexities. We apply that method to our (and to the sinusoidal, rotary and T5 bias) models in Appendix Table~\ref{tab:wt103_test_sota}. We find that our L = 3072 model surpasses the performance of Transformer-XL~\citep{transformer-xl}, the Sandwich~\citep{sandwich}, and Shortformer~\citep{shortformer} models. Our results are similar to the ones obtained with staged training~\citep{shortformer} but fall short of results obtained by Routing Transformer~\citep{roy2020efficient} and kNN-LM~\citep{khandelwal20generalization}. The methods used in those models are orthogonal to ours, and we hypothesize that combining them with ours might lead to even larger performance increases. 

After developing our method on WikiText-103, in Appendix Section~\ref{subsec:tbc}, we run one set of experiments on a  different domain (books) using a similar model architecture and without modifying any of the \al hyperparameters (the slopes) and show that our results fully transfer to this new domain. Our models are able to both surpass the sinusoidal baseline when not extrapolating while also outperforming it when extrapolating to longer sequences. 

\subsection{Results on the CC100+RoBERTa Corpus}

Our final set of experiments investigates whether \al transfers to a larger model trained with a larger computational budget on a larger dataset than the ones we previously used. We show that our method achieves strong results in this more challenging setting, obtaining similar performance to the sinusoidal baseline while using significantly less memory, since we train on shorter subsequences. 

The dataset we choose is a combination of the datasets used to train the RoBERTa~\citep{roberta} implementation of BERT~\citep{bert} and the English part of the CC-100 corpus introduced in~\cite{cc-100}, for a total of 461 GB. The RoBERTa training corpus---i.e., the Toronto Book Corpus~\citep{zhu2015aligning}, English Wikipedia, CC-News~\citep{ccnews}, OpenWebText~\citep{openwebtext} and Stories~\citep{stories})---is 161 gigabytes, and the English part of the CC-100 corpus is 300 gigabytes. 
The validation set contains 649K tokens.

 

Our models for this dataset have 25 transformer layers with 16 heads and a dimension of 2048, with an 8192 hidden dimension of the feedforward sublayers.
These models have 1.3B parameters. 
We train our models for one epoch, which is 50k updates on 128 V100 GPUs.

\begin{figure}
\centering
\begin{subfigure}{.45\textwidth} %
  \centering
  \includegraphics[width=\linewidth]{figures/ccr-train-512.pdf}
 
\end{subfigure}%
\begin{subfigure}{.45\textwidth} %
  \centering
 \includegraphics[width=\linewidth]{figures/ccr-train-1k.pdf}
\end{subfigure}
\caption{ On the left (right), a 1.3B-parameter ALiBi model trained on 512 (1024) and evaluated on 1024 (2048) tokens during training, compared to the sinusoidal baseline trained on 1024 (2048) tokens. The \al models obtain strong results even though they use 6\%-11\% less memory since they train on shorter sequences. Appendix Table \ref{tab:ccr} shows memory use and end-of-training perplexities.}
\label{fig:ccr_train_all}
\end{figure}

In Figure~\ref{fig:ccr_train_all} (left), we compare the validation perplexity for \li = 1024 throughout the training process for an \al model trained with \lt = 512 compared to the sinusoidal model trained with \lt = 1024. Since our model is trained on shorter sequences, it is 7\% faster and uses 1.6 GB less memory. We halt training of the sinusoidal baseline when our model reaches the end of its training (one epoch). 
At that time, our model is just 0.06 perplexity away from the baseline even though it was trained on sequences that are half the length of those the baseline used and requires less memory.

In Figure~\ref{fig:ccr_train_all} (right), results become even more impressive, showing that our model trained on \lt = 1024 outperforms by 0.09 perplexity the sinusoidal model trained on \lt = 2048 (when evaluating with \li = 2048) even though our model uses 3.1 GB less memory. Our model maintains a lead in perplexity over the sinusoidal model during the entire training process. By sampling five evenly distributed points across the training process, we compute that our \lt = 1024 model reaches a given perplexity value, on average, 11\% faster than the sinusoidal model does. 

Since our models in these comparisons use much less memory, they allow for stacking more layers, which would further improve performance (with negligible, if any, runtime cost). To keep our experiments as straightforward as possible, however, we do not add layers to our models.

Appendix Table~\ref{tab:ccr_all_50k} presents additional results comparing our models to the sinusoidal baseline when both are trained on the same $L$, showing that \al performs similarly to the sinusoidal baseline when not extrapolating. This contrasts with the  results presented on the smaller datasets, where \al consistently outperforms other position methods even when not extrapolating, suggesting that ALiBi's inductive bias provides additional benefits for lower-resource language modeling.

\begin{figure}[h]
\centering
\begin{subfigure}{.45\textwidth}  %
  \centering
  \includegraphics[width=\linewidth]{figures/ccr_extra_512.pdf}
\end{subfigure}%
\begin{subfigure}{.45\textwidth} %
  \centering
  \includegraphics[width=\linewidth]{figures/ccr_extra_1024.pdf}
\end{subfigure}
\caption{The ALiBi and sinusoidal models (with both $L$ = 512 and 1024) trained for 50k updates (1 epoch) on the CC100+RoBERTa corpus, extrapolating on the validation set. %
\al achieves the best results at around $2L$ but maintains strong performance even up to 10000 tokens in these experiments.}
\label{fig:ccr_extra}
\end{figure}

Figure~\ref{fig:ccr_extra} shows that our models trained on $L=512$ and $L=1024$ achieve the best results when extrapolating to about double the tokens that they were trained on.
Specifically, the \lt = 512 model (that obtains 9.79 perplexity when \li = 512) achieves its best score (9.3) when extrapolating to 1012 tokens, and the \lt = 1024 model (that obtains 9.16 perplexity when \li = 1024) achieves its best score (8.9) when extrapolating to 2024 tokens. 

One possible explanation is that the subsequences the model observes during training are up to $L$ tokens long. When performing inference on subsequences of length $2L$, half of the subsequences the model consumes are as long as the examples seen during training. When inference is performed on subsequences of length $2L+1$ or longer, less than half of the predictions the model makes are on subsequences of lengths seen during training, and that might degrade performance. 

The sinusoidal model cannot extrapolate at all in this setting, with its performance degrading for both the \lt = 512 and 1024 models as soon as one token more than $L$ is added during evaluation. 

In Appendix \ref{sec:analysis}, we find that \al's  edge over sinusoidal embeddings is largely explained by its improved avoidance of the early token curse.  We posit  that future work building on \al might achieve further gains by more efficiently  exploiting longer histories.

\section{Related Work}
In parallel with our work,~\citet{wennberg2021case} introduce a relative position method that, like our method, adds a bias to attention scores that is a function of the distance between the key and query elements. 
Unlike our \al method, which uses a  non-learned linear function, their method uses a radial-basis function, with multiple trainable parameters (in our experiments, this led to a slight decrease in runtime). 
In addition, they present experiments on text classification, not on language modeling.  They do not explore extrapolation.
The Distance Aware Transformer~\citep{da-transformer} multiplies attention scores by a bias that is a function of the distance between the key and query. This function uses a different, learned parameter in every head. They show results only on text classification. In our experiments (not presented), multiplying attention scores by the bias (instead of adding, as in \al) degraded performance. 

Transformer-XL~\citep{transformer-xl}  presented a language model that uses a cache and can attend to more tokens during inference than it was trained on (by increasing the length of the cache). However, this work presents results only where output length is limited to the $L$ (the training length), and their relative position method is very slow~\citep{shortformer}. 
The Longformer~\citep{longformer} adapts models trained on shorter sequences to document-level tasks. However, to achieve this they had to partially train their models on longer sequences. Our \al method enables extrapolation without any additional training on longer sequences. 

To our knowledge, extrapolation has not been previously explored in transformer language modeling, but it has been investigated previously and concurrently with transformers on other tasks, such as %
machine translation~\citep{rosendahl2019:pos_enc, neishi-yoshinaga-2019-relation, newman2020extrapolation, Kiyono2021SHAPESA}, sequence-to-sequence models trained on an artificial dataset~\citep{Hupkes2020}, pretrained sequence-to-sequence models tested on arithmetic tasks~\citep[Appendix C]{Nogueira2021InvestigatingTL}, models trained with reinforcement learning~\citep{lampinen2021towards}, image, speech recognition, and machine translation models~\citep{likhomanenko2021cape}, and protein structure prediction~\citep[Appendix 1.5]{Jumper2021HighlyAP}. 

\section{Conclusion}

We showed that the sinusoidal position embedding approach does not enable transformers to extrapolate to inputs longer than the ones they were trained on. We then established that extrapolation in transformers can be enabled by just changing the position method.
We showed that our \al method offers an extremely simple replacement for existing position approaches and allow models to extrapolate. In addition, when not extrapolating, our method achieves either better perplexity than the sinusoidal method (in models smaller than 1B parameters, trained on less data) or similar perplexity (in larger, billion parameter models trained on much more data). 
\al is simple to implement and does not slow down runtime or require extra parameters (but does occasionally require a negligible amount of extra memory).
Using our method, we sped up the training of a 1.3 billion parameter model evaluated on the same input sequence length as GPT-3 (2048). 

\subsubsection*{Acknowledgments}
We thank Tim Dettmers, Gabriel Ilharco, Jungo Kasai, Hao Peng, Sewon Min, Sofia Serrano, Sam Shleifer, Luke Zettlemoyer, Julian Michael, Nikolaos Pappas, Yizhong Wang, and the anonymous reviewers for their valuable feedback and fruitful discussions.
\clearpage

\clearpage

\end{document}