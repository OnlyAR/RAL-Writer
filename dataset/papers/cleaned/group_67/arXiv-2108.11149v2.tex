\title{A Unified Taxonomy and Multimodal Dataset for Events in Invasion Games}

\begin{document}

\title{A Unified Taxonomy and Multimodal Dataset for Events in Invasion Games}

\author{Henrik Biermann}
\authornote{Both authors contributed equally to this research.}
\email{h.biermann@dshs-koeln.de}
\orcid{0000-0001-5660-9876}
\affiliation{%
  \institution{Institute of Exercise Training and Sport Informatics, German Sport University Cologne}
  \streetaddress{}
  \city{Cologne}
  \state{}
  \country{Germany}
  \postcode{}
}

\author{Jonas Theiner}
\authornotemark[1]
\orcid{0000-0002-8966-4860}
\email{theiner@l3s.de}
\affiliation{
  \institution{L3S Research Center, Leibniz University Hannover}
  \city{Hannover}
  \state{}
  \country{Germany}
  \postcode{}
}

\author{Manuel Bassek}
\orcid{0000-0002-9394-913X}
\email{m.bassek@dshs-koeln.de}
\affiliation{%
  \institution{Institute of Exercise Training and Sport Informatics, German Sport University Cologne}
  \streetaddress{}
  \city{Cologne}
  \state{}
  \country{Germany}
  \postcode{}
}

\author{Dominik Raabe}
\email{dominik.raabe@dshs-koeln.de}
\orcid{0000-0001-7264-4575}
\affiliation{%
  \institution{Institute of Exercise Training and Sport Informatics, German Sport University Cologne}
  \streetaddress{}
  \city{Cologne}
  \state{}
  \country{Germany}
  \postcode{}
}

\author{Daniel Memmert}
\orcid{0000-0002-3406-9175}
\email{d.memmert@dshs-koeln.de}
\affiliation{%
  \institution{Institute of Exercise Training and Sport Informatics, German Sport University Cologne}
  \streetaddress{}
  \city{Cologne}
  \state{}
  \country{Germany}
  \postcode{}
}

\author{Ralph Ewerth}
\orcid{0000-0003-0918-6297}
\email{ralph.ewerth@tib.eu}
\affiliation{
  \institution{L3S Research Center \\ TIB -- Leibniz Information Centre for Science and
Technology}
  \city{Hannover}
  \state{}
  \country{Germany}
  \postcode{}
}

\renewcommand{\shortauthors}{Biermann and Theiner, et al.}

\begin{abstract}
The automatic detection of events in complex sports games like soccer and handball using positional or video data is of large interest in research and industry.
One requirement is a fundamental understanding of underlying concepts, i.e., events that occur on the pitch.
Previous work often deals only with so-called low-level events based on well-defined rules such as free kicks, free throws, or goals.
High-level events, such as passes, are less frequently approached due to a lack of consistent definitions. This introduces a level of ambiguity that necessities careful validation when regarding event annotations.
Yet, this validation step is usually neglected as the majority of studies adopt annotations from commercial providers on private datasets of unknown quality and focuses on soccer only.
To address these issues, 
we present~(1)~a universal taxonomy that covers a wide range of \textit{low} and \textit{high-level} events for invasion games and is exemplarily refined to soccer and handball,
and~(2)~release two multi-modal datasets comprising video and positional data with gold-standard annotations to foster research in fine-grained and ball-centered event spotting. 
Experiments on human performance demonstrate the robustness of the proposed taxonomy, and that disagreements and ambiguities in the annotation increase with the complexity of the event.
An I3D model for video classification is adopted for event spotting and reveals the potential for benchmarking.
Datasets are available at: \url{https://github.com/mm4spa/eigd}
\end{abstract}

\maketitle

\section{Introduction}\label{sec:intro}

\begin{figure}[bt!]
	\centering
	\includegraphics[width=\linewidth]{img/annotation_examples.pdf}
	\caption{\acrshort{dataset_provider}: Pass annotations from a \textcolor{set1_red}{data provider} vs. an \textcolor{set1_blue}{expert}. \acrshort{dataset_soccer} and \acrshort{dataset_handball}: Example annotations from experienced annotators~(\textcolor{set1_red}{red}, \textcolor{set1_blue}{blue}, \textcolor{set1_green}{green}) using our proposed taxonomy: Despite uncertainties regarding the concrete event type, the annotated timestamp often aligns. The mapping back to shared characteristics such as the motoric skill (e.g., ball release), leads to higher levels of agreement. 
	}
	\label{fig:example_annotations}
\end{figure}

Events play an important role for the (automatic) interpretation of complex invasion games like soccer, handball, hockey, or basketball.
Over the last years, three fundamental perspectives emerged with regard to the analysis of sports games, which all value different characteristics of the respective sports: 
(1)~The \emph{sports science} domain demands semantically precise descriptions of the individual developments to analyze success factors~\cite{lamas2014invasion}.
(2)~The \emph{machine learning} community aims to find automatic solutions for specific tasks~(often supervised).
(3)~\emph{Practitioners}~, i.e., coaches or analysts, show little interest in the description of the sport since they are rather interested in the immediate impact of specific modification of training or tactics. 
While the general objective to understand and exploit the underlying concepts in the sports is common to all perspectives, synergistic effects are barely observed~\cite{rein2016big}.
Descriptive statistics such as possession or shot frequency rely on events that occur on the pitch.
However, collecting semantic and (spatio-)~temporal properties for events during matches is non-trivial, highly dependent on the underlying definitions, and is, in the case of (accurate)~manual annotations, very time-consuming and expensive~\cite{pappalardo2019public}.
While it is a common practice of data providers~\cite{wyscout, opta, stats} for (certain)~matches in professional sport to delegate the annotation of events to human annotators, 
various approaches have been suggested to automate the process.
In this respect, the automatic detection of (spatio-)~temporal events has been addressed for (broadcast)~video data~\cite{giancola2018soccernet, giancola2021temporally, sarkar2019generation, vats2020event, sanford2020group, sorano2020automatic, hu2020hfnet, yu2019soccer, jiang2016automatic, liu2017soccer, tomei2021rms, karimi2021soccer, mahaseni2021spotting} and positional data~\cite{sanford2020group, xie2020passvizor, khaustov2020recognizing, chacoma2020modeling, richly2016recognizing, richly2017utilizing, morra2020slicing}.

The \textit{temporal event} localization is the task of predicting a semantic label of an event and assigning its start and end time, commonly approached in the domain of video understanding~\cite{caba2015activitynet}.  
Despite a general success in other domains~\cite{lin2019bmn, feichtenhofer2019slowfast, caba2015activitynet, nguyen2018weakly}, it has already been observed that this definition can lead to ambiguous boundaries~\cite{sigurdsson2017actions}.
Sports events can also be characterized by a single representative time stamp~(\emph{event spotting}~\cite{giancola2018soccernet}) and recently there has been success in spotting \textit{low-level} events~\cite{giancola2021temporally, deliege2020soccernet, cioppa2020context} in soccer videos such as goals and cards. 
In contrast, these data acquisition approaches lack more complex, ambiguous, and more frequent events like passes or dribblings that are not covered by existing publicly available~(video) datasets~\cite{feng2020sset, deliege2020soccernet}. Indeed, some definitions of \textit{high-level} events in soccer are provided in the literature~\cite{kim2019attacking, fernandes2019design}, but there is no global annotation scheme or even taxonomy that covers various events that can be evaluated with few meaningful metrics.
Although there are related events in other invasion games such as handball, neither a set of \textit{low-level} and \textit{high-level} events nor a taxonomy are defined in this domain.

A shared property for both tasks~(spotting and localization with start and end), regardless of the underlying event complexity, event property (temporal, spatial, or semantic), or data modality~(video or positional data), is the need for labeled event datasets to train and especially to evaluate machine learning approaches.
It is common to integrate~\cite{sanford2020group, fernandez2020soccermap} private event data from data-providers~(e.g., from \cite{wyscout, opta, stats}) of unknown~\cite{liu2013reliability} or moderate~(Figure~\ref{fig:example_annotations}~\acrshort{dataset_provider} as an example) quality.

In summary, we observe a lack of a common consensus for the majority of events in the sport.
Neither precise definitions of individual events nor the temporal annotation or evaluation process are consistent. 
Publicly available datasets are uni-modal, focus on soccer, and often consider only a small subset of events that does not reflect the entire match.
These inconsistencies make it for all aforementioned three perspectives difficult to assess the performance of automatic systems and to identify state-of-the-art approaches for the real-world task of fine-grained and ball-centered event spotting from multimodal data sources.

In this paper, we target the aforementioned problems and present several contributions: 1) We propose a unified taxonomy for \textit{low-level}, and \textit{high-level} ball-centered events in invasion games and exemplary refine it to the specific requirements of soccer and handball. This is practicable as most invasion games involve various shared motoric tasks~(e.g., a ball catch), which are fundamental to describe semantic concepts~(involving intention and context from the game).
Hence, it incorporates various base events relating to \textit{game status}, \textit{ball possession}, \textit{ball release}, and \textit{ball reception}.
2) We release two multimodal benchmark datasets~(video and audio data for soccer~(\acrshort{dataset_soccer}), synchronized video, audio, and positional data for handball~(\acrshort{dataset_handball})) with gold-standard event annotations for a total of 125 minutes of playing time per dataset.
These datasets contain frame-accurate manual annotations by domain experts performed on the videos based on the proposed taxonomy~(see Figure~\ref{fig:example_annotations}).
In addition, appropriate metrics suitable for both benchmarking and useful interpretation of the results are reported.
Experiments on the human performance show the strengths of the \textit{hierarchical} structure, the successful applicability to two invasion games, and reveal the expected performance of automatic models for certain events.
With the increasing complexity of an event~(generally deeper in the \textit{hierarchy}), ambiguous and differing subjective judgments in the annotation process increases.
A case study demonstrates that the annotations from data providers should be reviewed carefully depending on the application.
3) Lastly an \emph{I3D}~\cite{carreira2017quo} model for video chunk classification is adapted for the spotting task using a sliding window and non-maximum suppression and is applied.

The remainder of this paper is organized as follows. 
In Section~\ref{sec:rw}, existing definitions for several events and publicly available datasets are reviewed. The proposed universal taxonomy is presented in Section~\ref{sec:taxonomy}.
Section~\ref{sec:datasets} contains a description of the creation of the datasets along with the definition of evaluation metrics, while Section~\ref{sec:experiments} evaluates the proposed taxonomy, datasets, and baseline concerning annotation quality and uncertainty of specific events. 
Section~\ref{sec:conclusion} concludes the paper and outlines areas of future work.

\section{Related Work}\label{sec:rw}

We discuss related work on events in invasion games~(Section~\ref{rw:event_types}) and review existing datasets~(Section~\ref{rw:datasets}).

\subsection{Events Covered in Various Invasion Games}\label{rw:event_types} % Event Taxonomy

Common movement patterns have been identified in the analysis of spatio-temporal data~\cite{dodge2008towards} such as concurrence or coincidence.
While these concepts are generally applicable to invasion games, % in this work, 
our taxonomy and datasets focus on single actions of individuals~(players), which do not require a complex description of~(team)~movement patterns.
For the sport of handball, there are rarely studies on the description of game situations.
However, the influence of commonly understood concepts, such as shots and rebounds has been investigated~\cite{burger2013analysis}.
In contrast, for soccer, the description of specific game situations has been approached. \citet{kim2019attacking} focus on the attacking process in soccer.
\citet{fernandes2019design} introduce an observational instrument for defensive possessions. The detailed annotation scheme includes 14 criteria with 106 categories %(also considering situational variables, e.g., \textit{game status}, opponent quality, and match location) 
and achieved sufficient agreement in expert studies. However, the obtained semantic description and subjective rating of defensive possessions largely differ from our fine-grained objective approach. 
A common practice for soccer matches in top-flight leagues is to (manually) capture \textit{event data}~\cite{opta, pappalardo2019public}. 
The acquired data describe the on-ball events on the pitch in terms of soccer-specific events with individual attributes. 
While, in general, the inter-annotator agreement for % the modality 
this kind of data has been validated~\cite{liu2013reliability}, especially the \textit{high-level} descriptions of events are prone to errors. 
\citet{deliege2020soccernet} consider 17 well-defined categories which describe meta events, on-ball events, and semantic events during a soccer match. However, due to the focus of understanding a holistic video rather than a played soccer match, only 4 of the 17 event types describe on-ball actions, while more complex events, i.e., passes, are not considered. 
\citet{sanford2020group} spot \textit{passes}, \textit{shots}, and \textit{receptions} in soccer using both positional and video data. However, no information regarding definitions and labels is provided.

\subsection{Datasets}\label{rw:datasets}

To the best of our knowledge, there is no publicly available real-world dataset including positional data, video, and corresponding events, not to mention shared events across several sports.
The majority of datasets for event detection rely on video data and an individual sport domain.
In this context, \emph{SoccerNetV2}~\cite{deliege2020soccernet, giancola2018soccernet} was released, which is a large-scale action spotting dataset for soccer videos. %~(17 classes).
However, the focus is on spotting general and rarely occurring events such as \textit{goals}, \textit{shots}, or cards.
\emph{SoccerDB}~\cite{jiang2020soccerdb} and \emph{SSET}~\cite{feng2020sset} cover a similar set of general events. Even though they relate the events to well-defined soccer rules, they only annotate temporal boundaries.
\citet{pappalardo2019public} present a large event dataset, but it lacks definitions of individual events or any other data such as associated videos.
For basketball, \citet{Ramanathan_2016_CVPR} generated a dataset comprising five types of \textit{shots}, their related outcome (successful), and the \emph{steal event} by using Amazon Mechanical Turk. Here, the annotators were asked to identify the end-point of these events since the definition of the start-point is not clear. 
The \emph{SoccER} dataset~\cite{morra2020soccer} contains synthetically generated data~(positional data, video, and events) from a game engine. 
The volleyball dataset~\cite{ibrahim2016hierarchical} contains short clips with eight group activity labels such as right set or right spike where the center frame of each clip is annotated with per-player actions like standing or blocking. 

To summarize Section~\ref{rw:event_types} and~\ref{rw:datasets}, many studies consider only a subset of relevant~(\textit{low} and \textit{high-level}) events to describe a match.
The quality of both unavailable and available datasets is limited due to missing general definitions~(even spotting vs. duration) apart from well-defined~(per rule) events.

\section{General Taxonomy Design}\label{sec:taxonomy}

\begin{figure*}[tbh]
	\centering
	\includegraphics[width=\textwidth]{img/general_taxonomy_with_features.pdf}
	\caption{Base taxonomy for invasion games and example refinements for soccer and handball. Starting with basic motoric \emph{individual ball events}, the finer the hierarchy level, the semantic and necessary context information increases.} 
	\label{fig:taxonomy}
\end{figure*}

In this section, we construct a unified taxonomy for invasion games that can be refined for individual sports and requirements~(Figure~\ref{fig:taxonomy}). 
Initially, targeted sports and background from a sports science perspective are presented in Section~\ref{subsec:sports}. Preliminaries and requirements for our general taxonomy are listed in Section~\ref{subsec:characteristics}.
Finally, the proposed taxonomy, including concrete event types and design decisions, is addressed in Section~\ref{subsec:categories}.

\subsection{Targeted Sports \& Background}\label{subsec:sports}
Sports games share common characteristics and can be categorized in groups~(\emph{family resemblances})~\cite{Wittgenstein1999}.
Based on that idea,~\cite{Read1997, Hughes2002} structured sports games into three families: (1)~Net and wall games, which are score dependent (e.g., tennis, squash, volleyball), (2)~striking/fielding games, which are innings dependent (e.g., cricket, baseball), and (3)~invasion games, which are time-dependent (e.g., soccer, handball, basketball, rugby). 
In this paper, we focus on the latter. 
Invasion games all share a variation of the same objective: to send or carry an object (e.g., ball, frisbee, puck) to a specific target (e.g., in-goal or basket) and prevent the opposing team from reaching the same goal~\cite{Read1997}. The team that reaches that goal more often in a given time wins. Hence, we argue that the structure of our taxonomy can be applied to all invasion games with a sport-specific refinement of the base events. Please note that we refer to the object in the remainder of this work as a ball for clarity. 

Basic motor skills required in all invasion games involve controlled receiving of, traveling with, and sending of the ball~\cite{Roth2015}, as well as intercepting the ball and challenging the player in possession~\cite{Read1997}. 
Although different invasion games use different handling techniques, they all share the ball as an underlying characteristic. 
Thus, we find that ball events are central for describing invasion games. Moreover, since complex sport-science-specific events such as counterattack, possession play, tactical fouls, or any group activities like pressing are rather sports-specific, we focus on on-ball-ball events in this paper and refer to non-on-ball events as future work.  

\subsection{Characteristics \& Unification of Perspective}\label{subsec:characteristics}

We iteratively design the base taxonomy for invasion games to meet certain standards. To provide insights into this process, the following section details the underlying objectives.  

\paragraph{Characteristics}
For the design of a unified taxonomy for invasion games, we view specific characteristics as favorable.
~(1)~A \textit{hierarchical} architecture, in general, is a prerequisite for a clear, holistic structure. We aim to incorporate a format that represents a broad (general) description of events at the highest level and increases in degree of detail when moving downwards in the \textit{hierarchy}. This enables, i.e., an uncomplicated integration of individual annotations with varying degrees of detail as different annotated events (e.g., \textit{shot} and \textit{pass}) can fall back on their common property (here \emph{intentional ball release}) during evaluation.
However, please note that there exists no cross-relation in the degree of detail between different paths~(colors in Figure~\ref{fig:taxonomy}). Events from the same \textit{hierarchical} level may obtain different degrees of detail when from different paths.
~(2)~We target our taxonomy to be \textit{minimal} and \textit{non-redundant} since these characteristics require individual categories to be well-defined and clearly distinguishable from others. In this context, a specific event in the match should not relate to more than one annotation category to support a clear, unambiguous description of the match.
~(3)~The taxonomy needs to enable an \emph{exact} description of the match. While the previously discussed \textit{minimal}, \textit{non-redundant} design is generally important, an overly focus on these properties may disallow the description of the \textit{exact} developments in a match. Thus, any neglecting or aggregation for individual categories is carefully considered in the design of the taxonomy.
~(4)~Finally, we aim for a \textit{modular expendable} taxonomy. This allows for a detailed examination of specific sports and concepts while still ensuring a globally valid annotation that is comparable (and compatible) with annotations regarding different sports and concepts. 

\paragraph{Unification of Perspectives}
The targeted invasion games can generally be perceived from a variety of different perspectives. A mathematical view corresponds to a description of moving objects (players and the ball) with occasional stoppage and object resets (set-pieces reset the ball). On the other hand, a sport-scientist view interprets more complex concepts such as the mechanics of different actions or the semantics of specific situations of play.      
To unify these perceptions into a global concept, different approaches such as the \emph{SportsML} language~\cite{SportsML} or \emph{SPADL}~\cite{decroos2019actions} previously targeted a universal description of the match. However, given that the formats originate from a journalist perspective~\cite{SportsML} or provide an integration tool~\cite{decroos2019actions} for data from event providers~(see Section~\ref{exp:data_prov_quality}), they do not pursue the definition of precise annotation guidelines.
In contrast, we aim to provide a universal and \textit{hierarchical} base taxonomy that can be utilized by different groups and communities for the targeted invasion games.

\subsection{Annotation Categories}\label{subsec:categories}
The iteratively derived base taxonomy for invasion games is illustrated in Figure~\ref{fig:taxonomy}. Its %individual 
categories and attributes comply with the previously discussed characteristics~(see Section~\ref{subsec:characteristics}) and are outlined in this section~(see Appendix for more detailed definitions).

\subsubsection{Game Status Changing Event}
We initialize the first path in our \textit{base taxonomy} such that it corresponds with the most elemental properties of invasion games. Thus, we avoid integrating any semantic information (tactics, mechanics) and regard the so-called, \textit{game status} which follows fixed game rules~\cite{IFAB, IHF}.
The \textit{game status} provides a deterministic division of any point in the match into either active (running) on inactive (paused) play. %(also referred to as the activeness). 
In the sense of a \textit{minimal} taxonomy, we find that an \textit{exact} description of the current \textit{game status} is implicitly included by annotating only those events which cause changes to the current \textit{game status}~(see yellow fields in Figure~\ref{fig:taxonomy}).
Moreover, in all targeted invasion games, a shift of the \textit{game status} from active to inactive only occurs along with a rule-based \textit{referee's decision} (foul, ball moving out-of-bounds, game end, or sport-specific stoppage of play) while a shift from inactive to active only occurs along \textit{static-ball-action} (game start, ball in field, after foul, or sport-specific resumption of play). Thus, we discriminate between these two specifications in the path and maintain this \textit{hierarchical} structure.

\subsubsection{Ball Possession}
The following paths in our taxonomy comprise additional semantic context to enable a more detailed assessment of individual actions and situations. In this regard, we consider the concept of \textit{possession} (see purple field in Figure~\ref{fig:taxonomy}) as defined by~\citet{link2017individual}. Albeit generally not included in the set of rules of all targeted invasion games (exceptions, i.e., for basketball), the assignment of a team's \textit{possession} is a common practice, and its importance is indicated, for instance, by the large focus of the sports science community~\cite{camerino2012dynamics, casal2017possession, jones2004possession, lago2010game}. 
Similar to the \textit{game status}, we only consider the changes to the \textit{possession} with respect to a \textit{minimal} design. 

\subsubsection{Individual Ball Events}

Related to the concept of individual ball \textit{possession} are \textit{individual ball events}, defined as events within the sphere of an individual \textit{possession}~\cite{link2017individual}~(see green fields in Figure~\ref{fig:taxonomy}). 
Along with the definition for an individual \textit{possession}, \citet{link2017individual} define individual \textit{ball control} as a concept requiring a certain amount of motoric skill.
This also involves a specific start and end time for \textit{ball control} which already enables a more precise examination of \textit{individual ball event}.

At the start point of individual \textit{possession}, the respective player gains (some degree of) \textit{ball control}. We refer to this moment as a \textit{ball reception}, describing the motoric skill of gaining \textit{ball control}.
Analogously, at the endpoint of \textit{possession}, the respective player loses \textit{ball control}. We specify this situation as a \textit{ball release}, independent of the related intention or underlying cause. Please note that for a case where a player only takes one touch during an individual \textit{ball control}, we only consider the \textit{ball release} as relevant for the description.
For time span between \textit{ball reception} and \textit{ball release}, in general, the semantic concept of a \textit{dribbling} applies. However, various definitions for (different types of) \textit{dribbling} depend on external factors such as the sport, the context, and the perspective. As this semantic ambiguity prevents an \textit{exact} annotation, we do not list \textit{dribbling} as a separate category in the taxonomy but refer this concept to sport-specific refinements.     

At this point, we utilized the two concepts \textit{game status} and \textit{possession} in invasion games to design the initial \textit{hierarchical} levels for a total of three different paths within the taxonomy~(yellow boxes, purple boxes, and two highest \textit{hierarchical} levels of the green boxes). Accordingly, the required amount of semantic information for the presented levels is within these two concepts. Moreover, since an assessment of these concepts requires low semantic information, we think that the current representation is well-suited for providing annotations in close proximity to the previously presented mathematical perspective on the match. 

However, we aim for a step-wise approach towards the previously presented sport-scientist perspective for the subsequent \textit{hierarchical} levels.
Therefore, we increase the amount of semantic interpretation by regarding additional concepts, i.e., the overall context within a situation or the intention of players.
To this end, we distinguish between two different subcategories for a \textit{ball release}: \textit{intentional} or \textit{unintentional ball release}.
Regarding an \textit{unintentional ball release}, we generally decide between the categories \textit{successful interference}~(from an opposing player) and \textit{self-induced}~(describing a loss of \textit{ball control} without direct influence of an opponent).
\newpage
In contrast, for an \textit{intentional ball release}, we further assess the underlying intention or objective of a respective event. We discriminate between a \textit{pass}, including the intention that a teammate receives the released ball, and a \textit{shot} related with an intention (drawn) towards the target. 
In some rare cases, the assessment of this intention may be subjective and difficult to determine. However, specific rules in invasion games require such assessment, i.e., in soccer, the goalkeeper is not allowed to pick up a released ball from a teammate when it is "\emph{deliberately}" kicked towards him~\cite{IFAB}.

Please note that we define \textit{individual ball events} as mutually exclusive, i.e., only one event from that path can occur at a specific timestamp. However, a single point in time may generally include multiple events from different paths in the \textit{taxonomy}. 
Examples for this, in particular, are set-pieces. Here, we annotated a \textit{static-ball event} such as \emph{ball in field}, indicating the previously discussed shift of the \textit{game status}, and an \textit{individual ball event} (e.g., a \textit{pass}) describing the concrete execution. This is necessary as the type of \textit{static-ball event} does not definitely determine the type of \textit{individual ball event}~(i.e., a free-throw in basketball can theoretically be played as a pass from the rim). Nevertheless, since each set piece involves some sort of \textit{ball release}~(per definition of rule, a set-piece permits a double contact of the executing player), an \textit{exact} annotation of set-pieces is provided by the implicit link of simultaneous~(or neighboring) \textit{ball release} and \textit{static ball events}.

\subsubsection{Attributes}
A global method to add semantic information to the annotation is provided by defining specific \textit{attributes} for certain events. 
While not representing a specific path in the \textit{base taxonomy}, an \textit{attribute} is defined as a name or description like \emph{pixel location} of the event in the video~(Figure~\ref{fig:taxonomy} upper-right) and thus provides additional information to the respective event.
When an \textit{attribute} is defined for an event at a certain \textit{hierarchical} level, it is valid for all child events in lower levels.

\section{Events in Invasion Games Dataset}\label{sec:datasets}

\begin{table}[b]
\caption{Dataset distribution: Approx. 40\,\% of all events are reserved for testing, i.e., two of five matches, respectively.}
\label{tab:stats}
\centering
\small
\fontsize{7}{10}\selectfont
\def\arraystretch{0.7}

\begin{tabularx}{\linewidth}{Xlrr}
\toprule
Shared Parent Event & Event &    \acrshort{dataset_soccer} &  \acrshort{dataset_handball} \\
\midrule \midrule
\textbf{ball possession change} &  &            171 &              136 \\ \hline
\textbf{ball reception} &  &           923 &            2268 \\ \hline
\textbf{ball release} &  &           1531 &            2470 \\ \hline
\textbf{pass} &  &            1346 &              2292 \\
 & intercepted &            83 &              14 \\
                           & off target &           175 &               9 \\
                           & successful deflected &            24 &               6 \\
                           & successful untouched &          1064 &            2263 \\
\textbf{shot} &  &            31 &              175 \\
                           & blocked/intercepted &   17 &              61 \\
                           & goal frame &             0 &               8 \\
                           & off target &             8 &              12 \\
                           & successful &             6 &              94 \\
\textbf{unintentional} & other &            74 &               0 \\
                           & successful interference &            80 &               3 \\ \hline
\textbf{referee decision}         &        &            142 &               252 \\ \hline
                           & ball out of field &           101 &              21 \\
                           & foul &            32 &             114 \\
                           & goal &             3 &              86 \\
                           & other &             5 &              10 \\
                           & two min &             n.d. &               8 \\
                           & yellow &             1 &              13 \\ \hline
\textbf{static ball action}         &         &            121 &               207 \\ \hline
                           & corner &            11 &               n.d. \\
                           & free-kick &            29 &              84 \\
                           & game start &             1 &               2 \\
                           & goal-kick &            18 &               n.d. \\
                           & kick-off &             1 &              87 \\
                           & other &             0 &               6 \\
                           & penalty &             1 &              11 \\
                           & throw-in &            60 &              17 \\
\bottomrule
\end{tabularx}

   

\end{table}
The following Section~\ref{sec:dataset_description} describes our multimodal (video, audio, and positional data) and multi-domain (handball and soccer) dataset for ball-centered event spotting~(\acrshort{dataset}). 
In Section~\ref{sec:metrics}, appropriate metrics for benchmarking are introduced.

\subsection{Data Source \& Description}\label{sec:dataset_description}
To allow a large amount of data diversity as well as a complete description of a match, we regard longer sequences from different matches and stadiums. 
We select 5 sequences à 5 minutes from 5 matches resulting in 125 minutes of raw data, respectively, for handball and soccer.

\paragraph{Data Source}
For the handball subset, referred as \acrshort{dataset_handball}, synchronized video and positional data from the first German league from 2019 are kindly provided by the Deutsche Handball Liga and Kinexon\footnote{\url{https://kinexon.com/}} and contain HD~($1280 \times 720$ pixels) videos at 30\,fps and positional data for all players in 20\,Hz.
The videos include unedited recordings of the game from the main camera~(i.e., no replays, close-ups, overlays, etc.).
Some events are more easily identified from positional data, other events that require visual features can be extracted from video, making \acrshort{dataset_handball} interesting for multimodal event detection.

For the soccer dataset, referred to as \acrshort{dataset_soccer}, we collect several publicly available broadcast recordings of matches from the FIFA World Cup (2014, 2018) in 25\,fps due to licensing limitations without positional data.
A characteristic for annotation purpose is naturally given regarding the utilized videos. \acrshort{dataset_soccer} includes difficulties for the annotation, such as varying camera angles, long replays, or the fact that not all players are always visible in the current angle, making it challenging to capture all events for a longer sequence.
All events are included that are either visible or can be directly inferred by contextual information~(e.g., timestamp of a \emph{ball release} is not visible due to a cut from close-up to the main camera). However, with exceptions for \emph{replays} as these do not reflect the actual time of the game.

\paragraph{Annotation Process \& Dataset Properties}\label{sec:dataset:annotation_process}

To obtain the dataset annotations, the general task is to spot the events in the lowest \textit{hierarchy} level since the parent events~(from higher \textit{hierarchy} levels) are implicitly annotated. Therefore, the taxonomy~(Section~\ref{sec:taxonomy}) and a concrete annotation guideline including definitions for each event, examples, and general hints~(see Appendix) were used.
An example for \emph{unintentional ball release - self-induced} is this situation: A player releases the ball without a directly involved opponent, e.g., slips/stumble or has no reaction time for a controlled \textit{ball release}. For instance, after an \textit{intercepted/blocked pass} or \textit{shot} event. Timestamp: on \textit{ball release}.

We hired nine annotators~(sports scientists, sports students, video analysts). 
Due to the complexity of soccer, four of them annotated each sequence of the \acrshort{dataset_soccer} test set.
Note, that two of the five matches are indicated as a test set~(\acrshort{dataset_soccer}-T, \acrshort{dataset_handball}-T), respectively.
In addition, one inexperienced person without a background in soccer annotated the \acrshort{dataset_soccer}-T.
For \acrshort{dataset_handball}, three experienced annotators also processed each sequence of the test set.
An experienced annotator has labeled the remaining data~(e.g., reserved for training).
The annotation time for a single video clip is about 30 minutes for both datasets.
The number of events given the entire dataset and one expert annotation is presented in Table~\ref{tab:stats} and we refer to Section~\ref{exp:aggreement} for the assessment of the annotation quality.
Figure~\ref{fig:example_annotations} shows two sequences with annotations from several persons as an example.

For each dataset, we assess the human performance~(see Section~\ref{exp:aggreement}) for each individual annotator. The annotation with the highest results is chosen for release and as reference in the evaluation of the baseline~(see Section~\ref{exp:baseline}).

\subsection{Metrics}\label{sec:metrics}

For events with a duration, it is generally accepted~\cite{lin2019bmn, caba2015activitynet} to measure the \acrfull{temporal_iou}. 
This metric computes for a pair of annotations by dividing the intersection~(overlap), by the union of the annotations. The \acrshort{temporal_iou} for multiple annotations is given by the ratio of aggregated intersection and union. 
Regarding events with a fixed timestamp, a comparison between annotations is introduced in terms of temporal tolerance. Thereupon, when given a predicted event, count a potentially corresponding ground-truth event within the respective class as a true-positive, if and only if it falls within a tolerance area~(in seconds or frames). 
Yet, the definition of corresponding events from two different annotations is non-trivial~\cite{sanford2020group, deliege2020soccernet, giancola2018soccernet}, especially for annotations with different numbers of annotated events. 
A common method to circumvent this task is to introduce a simplification step using a \acrfull{nnm} which considers events with the same description on the respective \textit{hierarchy} level. 
After defining true positive, false positive, and false negative, this enables the computation of the \acrfull{temporal_ap}~(given by the average over multiple temporal tolerance areas~\cite{deliege2020soccernet, giancola2018soccernet}) or precision and recall for a fixed temporal tolerance area~\cite{sanford2020group}.

However, as the \acrshort{nnm} generally allows many-to-one mappings, a positive bias is associated with it. For instance, when multiple events from a prediction are assigned to the same ground-truth event~(e.g., \textit{shot}), they might all be counted as true positives~(if within the tolerance area), whereas the mismatch in the number of events is not further punished. This bias is particularly problematic for automatic solutions that rely on~(unbiased) objectives for training and evaluation. 
Therefore, \citet{sanford2020group} apply a \acrfull{nms} which only allows for a single prediction within a respective \acrshort{nms} window. While this presents first step, a decision on the (hyper-parameter) \acrshort{nms} window length can be problematic. When chosen too large, the \acrshort{nms} does not allow for a correct prediction of temporally close events. In contrast, when chosen too small, the \acrshort{nms} only partially accounts for the issue at hand. Moreover, the lack of objectivity draws a hyper-parameter tuning, e.g., a grid search, towards favoring smaller window lengths for \acrshort{nms}. 

To avoid these issues, we propose an \emph{additional} method to establish a one-to-one mapping for corresponding events from two annotations~(with possibly different numbers of events). 
In theory, this mapping can only be established if the number of event types between the annotations is equal. However, in practice, this requirement is rarely fulfilled for the whole match. Moreover, even when fulfilled, possibly additional and missing events might cancel each other out. 
Based on this, a division of the match into independent~(comparable) segments is a reasonable pre-processing step. Thus, we define a \textit{sequence} as the time of an active match between two \textit{game status-changing events}~(objectively determined by the set of rules~\cite{IFAB, IHF}). Then, (i)~we count the number of \textit{sequences} in two \textit{annotations} to verify that no \textit{game status changing events} were missed~(and adopt \textit{game status changing events} that were missed), (ii)~count the number of annotated events of the same category within a \textit{sequence}, and (iii)~assign the corresponding events relative to the order of occurrence within the \textit{sequence} only if the number of annotations matches.
If this number does not match, we recommend to either separately consider the \textit{sequence} or to fully discard the included \textit{annotations}.
In analogy to \acrshort{nnm}, we refer to this method as \acrfull{scm}.
Please note that, relative to the degree of detail within the compared \textit{annotations}, the contained additional information (for example player identities) can be used to increase the degree of detail in \acrshort{scm}.

\section{Experiments}\label{sec:experiments}

\begin{table}[b!]
\caption{Measuring the temporal IoU of several annotators for events with a duration.}
\label{tab:tiou}
\centering
\setlength{\tabcolsep}{2pt}
\small
\fontsize{7}{10}\selectfont
\begin{tabularx}{\linewidth}{Xl|r|r}
\toprule
Annotation & Dataset &  Game Status     & Ball Possession   \\ \midrule                        
\multicolumn{1}{l|}{\multirow{2}{*}{Average Experienced}}               & \acrshort{dataset_handball}-T  &     $0.68 \pm 0.02$            &    $0.72 \pm 0.02$          \\
\multicolumn{1}{l|}{}                                           & \acrshort{dataset_soccer}-T  &   $ 0.92 \pm 0.01 $            &  $0.78 \pm 0.03$             \\
\multicolumn{1}{l|}{Inexperienced vs. Experienced}                                           &\acrshort{dataset_soccer}-T  &     $0.92 \pm 0.03$           & $0.73 \pm 0.05$  \\
\bottomrule
\end{tabularx}
\end{table}

\newcolumntype{R}[2]{%
    >{\adjustbox{angle=#1,lap=\width-(#2)}\bgroup}%
    l%
    <{\egroup}%
}
\newcommand*\rot{\multicolumn{1}{R{90}{1em}}}% no optional argument here, please!

\begin{table*}[tbh!]
\caption{Expected human and from the baseline achieved performance: Precision and Recall (in~$\%$) after applying Nearest Neighbour (NN) matching and the proposed Sequence Consistency (SC) matching for representative events at multiple hierarchy levels. Note, that an appropriate event-specific evaluation window length~$w_\text{eval}$~[s] is applied. The number of consistent events for SC matching (in~$\%$) are indicated in brackets. {\normalfont\footnotesize $^2$No \emph{ball out of field} annotations provided.}}
\label{tab:aggreement_nn_cm}
\centering
\setlength{\tabcolsep}{2.2pt}
\small
\fontsize{7}{10}\selectfont
\def\arraystretch{1.0} % 0.85
\begin{tabularx}{\textwidth}{l|l|llllllllllllllllll}
\toprule
                        & Event                    & \multicolumn{2}{l}{\emph{Status Change}}   & \multicolumn{2}{l}{\emph{Reception}}   & \multicolumn{2}{l}{\emph{Release}}   & \multicolumn{2}{l}{\emph{Int. Release}}   & \multicolumn{2}{l}{\emph{Unint. Release}}   & \multicolumn{2}{l}{\emph{Shot}}   & \multicolumn{2}{l}{\emph{Suc. Interf.}}   & \multicolumn{2}{l}{\emph{Suc. Pass}}   & \multicolumn{2}{l}{\emph{Interc. Pass}}   \\
Dataset                 & Matching                 & NN       & SC           & NN       & SC           & NN       & SC           & NN       & SC           & NN       & SC           & NN      & SC            & NN       & SC           & NN       & SC           & NN       & SC           \\ \hline 

\multirow{6}{*}{\acrshort{dataset_handball}-T} & $w_\text{eval}$            & \multicolumn{2}{c}{6.04} & \multicolumn{2}{c}{0.44} & \multicolumn{2}{c}{0.44} & \multicolumn{2}{c}{0.44} & \multicolumn{2}{c}{2.04} & \multicolumn{2}{c}{0.44} & \multicolumn{2}{c}{2.04} & \multicolumn{2}{c}{0.44} & \multicolumn{2}{c}{0.44} \\ \hline \midrule
& Num. Events               & \multicolumn{2}{c}{$135.7 \pm 23.0$}  & \multicolumn{2}{c}{$821.0 \pm 10.7$}  & \multicolumn{2}{c}{$844.7 \pm 26.8$}  & \multicolumn{2}{c}{$841.0 \pm 26.5$}  & \multicolumn{2}{c}{$3.7 \pm 0.5$}  & \multicolumn{2}{c}{$62.7 \pm 1.2$}  & \multicolumn{2}{c}{$3.0 \pm 0.8$}  & \multicolumn{2}{c}{$765.7 \pm 25.8$}  & \multicolumn{2}{c}{$6.7 \pm 0.9$}  \\ \cline{2-20}
                        & Mean Exp. (Prc.=Rec.)              
                        & 78.9         & 90.0 (40)
                        & 92.2         & 89.7 (45)
                        & 92.7         & 82.3 (33)
                        & 92.9         & 83.3 (33)
                        & 18.2         & 100 (18)
                        & 96.3         & 99.4 (93)
                        & 22.2         & 100 (22)
                        & 92.4         & 82.6 (37)    
                        & 45.0         & 100 (45)
                        \\
                        & Baseline vs. Exp. Prc.              
                        &\multicolumn{2}{c}{-}
                        & 45.6         &    0.0 (0)          
                        &  \multicolumn{2}{c}{-}           
                        &  \multicolumn{2}{c}{-}            
                        & \multicolumn{2}{c}{-}            
                        &  43.2        &   0.0 (0)            
                        &  \multicolumn{2}{c}{-}         
                        &  46.4        &    0.0 (0)          
                        &    \multicolumn{2}{c}{-} 
                        \\
                        & Baseline vs. Exp. Rec.              
                        & \multicolumn{2}{c}{-}
                        & 93.9         &    0.0 (0)          
                        &  \multicolumn{2}{c}{-}          
                        &  \multicolumn{2}{c}{-}            
                        &  \multicolumn{2}{c}{-}            
                        & 41.0         &   0.0 (0)            
                        & \multicolumn{2}{c}{-}           
                        & 91.5         &  0.0 (0)            
                        &    \multicolumn{2}{c}{-} 

                        
 \\ \hline \midrule
 
\multirow{4}{*}{\acrshort{dataset_soccer}-T}
& Num. Events               & 
                        \multicolumn{2}{c}{$113.4 \pm 3.4$ %(77)
                        } & \multicolumn{2}{c}{$362.6 \pm 5.6$ %(61)
                        } & \multicolumn{2}{c}{$550.2  \pm 13.4$ %(45)
                        }  & \multicolumn{2}{c}{$500.0  \pm 7.0$ %(49)
                        }  & \multicolumn{2}{c}{$50.2 \pm 14.2$ %(48)
                        }  & \multicolumn{2}{c}{$12.2  \pm 0.4$ %(98)
                        } & \multicolumn{2}{c}{$32.0 \pm 4.3$ %(56)
                        } & \multicolumn{2}{c}{$385.2 \pm 4.5$ %(54)
                        } & \multicolumn{2}{c}{$58.8  \pm 18.9$ %(38)$
                        }  \\ \cline{2-20}
                        & Mean Exp. (Prc.=Rec.)           
                        & 95.0 & 98.7 (78)
                        & 94.9 & 94.5 (61)
                        & 95.7 & 90.8 (42)
                        & 96.2 & 93.3 (48)
                        & 62.7 & 84.4 (48)
                        & 100.0 & 100.0 (100)
                        & 68.5 & 91.8 (57)
                        & 96.0 & 94.0 (54)
                        & 60.0 & 85.0 (33)
                        \\
                        
                        & Inexp. vs Exp. Prc.     
                        & 95.9 & 98.8 (77)
                        & 90.3 & 89.7 (61)
                        & 93.4 & 90.0 (49)
                        & 94.8 & 87.9 (51)
                        & 64.1 & 78.9 (49)
                        & 92.3 & 100 (100)
                        & 64.5 & 83.3 (55)
                        & 92.6 & 93.2 (53)
                        & 65.5 & 85.5 (44)
                        \\
                        & Inexp. vs Exp. Rec.     
                        & 93.8 & 98.8 (74)
                        & 88.9 & 89.6 (60)
                        & 92.5 & 89.5 (49)
                        & 93.2 & 87.0 (50)
                        & 60.0 & 78.9 (47)
                        & 100 & 100 (92)
                        & 62.0 & 83.3 (55)
                        & 94.3 & 92.5 (54)
                        & 71.4 & 85.5 (48)
                        \\ \hline \midrule 
                        \multirow{3}{*}{\acrshort{dataset_provider}}
                         & Num. Events & \multicolumn{2}{c}{$649.0 \pm 8.0$\tablefootnote{ }} 
                         & \multicolumn{2}{c}{-}  &  \multicolumn{2}{c}{-}  & \multicolumn{2}{c}{-}  & \multicolumn{2}{c}{-}  & \multicolumn{2}{c}{$100.5 \pm 2.5$}  & \multicolumn{2}{c}{-}  & \multicolumn{2}{c}{$2428.0 \pm 24.0$}  & \multicolumn{2}{c}{-}  \\
                        \cline{2-20}
                        & Data Provider Prc.
                        &  74.7 & 71.1 (96)          
                        &  \multicolumn{2}{c}{-} 
                        &  \multicolumn{2}{c}{-} 
                        &  \multicolumn{2}{c}{-}               
                        &  \multicolumn{2}{c}{-}               
                        &  6.8 & 8.0 (73)                
                        &  \multicolumn{2}{c}{-}               
                        & 12.9         & 12.3 (65)              
                        &  \multicolumn{2}{c}{-} 
                        \\
                        & Data Provider Rec.
                        &  73.5 & 71.1 (93)           
                        &  \multicolumn{2}{c}{-} 
                        &  \multicolumn{2}{c}{-} 
                        &  \multicolumn{2}{c}{-}               
                        &  \multicolumn{2}{c}{-}              
                        &  7.1 & 8.1 (76)               
                        &  \multicolumn{2}{c}{-}               
                        & 12.6         & 12.6 (65)              
                        &  \multicolumn{2}{c}{-}  
                    
 \\ \bottomrule
\end{tabularx}
\end{table*}

We assess the quality of our proposed dataset by measuring the expected human performance~(Section~\ref{exp:aggreement}) and present a baseline classifier that only utilizes visual features~(Section~\ref{exp:baseline}). 
The %uncertainty 
quality of annotations from an official data provider is evaluated in Section~\ref{exp:data_prov_quality}.

\subsection{Assessment of Human Performance}\label{exp:aggreement}

Despite we aim to provide as clear as possible definitions for the annotated events, the complex nature of invasion games might lead to uncertain decisions during the annotation process.
According to common practice, we assess the annotation quality and, hence, expected performance of automatic solutions by measuring the average human performance on several evaluation metrics~(Section~\ref{sec:metrics}). 
In this respect, one annotator is treated as a predictor and compared to each other annotator, respectively, considered as reference. 
Consequently, the average over all reference annotators represents the individual performance of one annotator while the average across all individual performances corresponds to the average human performance. We report the average performance for experienced annotators for \acrshort{dataset_handball}-T and \acrshort{dataset_soccer}-T while we additionally assess the generality of our taxonomy by comparing the individual performance of domain experts and an inexperienced annotator for \acrshort{dataset_soccer}-T.

For events with a duration (\emph{game status}, \emph{possession}), we report the \acrshort{temporal_iou}.
To evaluate the event spotting task, a sufficient assessment of human performance requires a multitude of metrics. Similar to~\citet{sanford2020group}, we report the precision and recall by applying the \acrshort{nnm} for individual events at different levels of our proposed \textit{hierarchy}. %However, in contrast to previous works, 
We define strict but meaningful tolerance areas for each event to support the general interpretability of the results. 
Additionally, we apply the \acrshort{scm} where we compensate for a possible varying number of sequences by adopting the sequence borders in case of a possible mismatch. 
We report precision and recall for events from consistent sequences along with the percentage of events from consistent sequences. The Appendix provides a detailed overview of each individual annotator performance.

\paragraph{Results \& Findings}

The overall results for events with a duration~(Table~\ref{tab:tiou}) and events with a timestamp~(Table~\ref{tab:aggreement_nn_cm}) indicate a general agreement for the discussed concepts.
Moreover, the minor discrepancies in the performance of the experienced and the inexperienced annotator for \acrshort{dataset_soccer}-T also indicate that a sufficient annotation of our base taxonomy does generally not require expert knowledge. This observation shows the low amount of semantic interpretation included in our proposed taxonomy. Please note that due to the asymmetry in the comparison (one inexperienced annotator as prediction and four experienced annotators as reference), for this case, the precision and recall differ in Table~\ref{tab:aggreement_nn_cm}.

In Table~\ref{tab:tiou}, the agreement for \textit{game status} in soccer is significantly higher than the agreement in \textit{possession}. For handball, while the results for \textit{possession} are comparable to soccer, the agreement for \textit{game status} is significantly lower. This likely originates from the rather fluent transitions between active and inactive play which complicate a clear recognition of \textit{game status change events} in handball.
In contrast, general similarities in the annotations for \acrshort{dataset_soccer}-T and \acrshort{dataset_handball}-T can be found in agreement for individual ball events~(Table~\ref{tab:aggreement_nn_cm}). Beneath the previously discussed differences in the ambiguity of \textit{game status}, reflected in inferior agreement of \textit{game status change events}, similar trends are observable in both sports~(limitations, i.e., for infrequent events in handball such as \textit{unintentional ball release} or \textit{successful interference}).
For both datasets, the \textit{hierarchical} structure positively influences the results where the highest level shows a high overall agreement which decreases when descending in the \textit{hierarchy}. This relates to the similarly increasing level of included semantic information~(see Section~\ref{subsec:characteristics}) complicating the annotation. However, this general observation does not translate to each particular event in the taxonomy. 

The results for \acrshort{scm} provide a valuable extension to the informative value of \acrshort{nnm}, i.e., to detect the positive bias~(Section~\ref{sec:metrics}). For instance, the \textit{successful pass} for \acrshort{dataset_handball}-T shows a general high agreement. However, a positive bias in this metric can be recognized regarding the comparatively low amount of sequence-consistent events~(in brackets). These differences are probably caused by the high frequency of \textit{successful passes} in handball and the connected issues with assignment, detailed in Section~\ref{sec:metrics}.  

Typical misclassifications are often related to the assignment of intention. For ambiguous situations~(see Figure~\ref{fig:example_annotations}), this assignment can be difficult and largely depending on the outcome. For instance, if a played ball lands close to a teammate the situation will rather be annotated as \textit{intentional ball release}. However, this does not comply with the concept of intention that needs to be distinguished in the moment of the execution. Yet, due to the complex nature of invasion games, even the player who played the ball might not give a definite answer.          
A different type of error are temporal mismatches~(such as delays). While generally excluded from the annotation, still, a common source for these temporal differences are cuts, replays, or close-ups in the video data. As we aim to include the majority of events
if the action on the pitch can be derived from the general situation~(i.e., a replay only overlaps with a small fraction of an event), a common source of error are different event times. This is especially relevant for \textit{game status change events} where cuts and replays commonly occur.

\subsection{Vision-based Baseline}\label{exp:baseline}
To present some potential outputs of an automated classifier model, we create a baseline that only uses visual features from the raw input video to spot events.
Due to the lack of existing end-to-end solutions for event spotting and density of events (approx. each second in \acrshort{dataset_handball}), we follow common practice, where first a classifier is trained on short clips and then a sliding window is applied to produce frame-wise output~(e.g., feature vectors or class probabilities). 
We follow \cite{sanford2020group} and directly apply \acrshort{nms} to the predicted class probabilities to suppress several positive predictions around the same event.

\subsubsection{Setup for Video Chunk Classification}
For the model, we choose an Inflated 3D ConvNet~\emph{I3D}~\cite{carreira2017quo, NonLocal2018} with a \emph{ResNet-50} as backbone which is pre-trained on \emph{Kinetics400}~\cite{kay2017kinetics}. 
We select three classes~(\emph{reception}, \emph{successful pass}, and \emph{shot})~(plus a background event). We train one model for \acrshort{dataset_handball} on the entire~(spatial) visual content with fixed input dimension of $Tx3x256x456$.
Short clips~($T=32$ frames), centered around the annotation, are created to cover temporal context. For the background event, all remaining clips are taken with a stride of 15 frames. 
Temporal resolution is halved to during training~(i.e., 15\,fps for \acrshort{dataset_handball}).
For remaining details we refer to the Appendix.
The model with the lowest validation loss is selected for the event spotting task.

\subsubsection{Evaluating the Event Spotting Task}
We collect all predicted probabilities at each frame using a sliding window and apply \acrshort{nms} on validated event-specific filter lengths~$w^e_\text{nms}$. 
As several events can occur at the same time, for each event~$e$ a confidence threshold~$\tau_e$ is estimated.
Both hyper-parameters are optimized for each event on the $F_1$ score with \acrshort{nnm} using a grid search on the training dataset. We use the same search space as \citet{sanford2020group}.

Results are reported in Table~\ref{tab:aggreement_nn_cm} where precision and recall are calculated considering the expert annotation with the highest human performance as ground-truth.
Despite the limited amount of training data, the baseline demonstrates that our proposed datasets are suitable for benchmarking on-ball events.
We qualitatively observe that an excessive number of positive predictions in spite of \acrshort{nms} causes bad performance using \acrshort{scm}, which is only partly visible when using \acrshort{nnm}. 
This confirms the need for the proposed metric and identifies the error cases of the baseline.
The model achieves sufficiently robust recognition performance with temporal centered ground-truth events, it predicts the actual event with high confidence when an ground-truth event in the sliding window is not centered.
We refer to future work~(1)~to improve the visual model for instance with hard-negative sample mining, or temporal pooling~\cite{giancola2021temporally} and~(2)~for the usage of multimodal data~(e.g., \cite{vanderplaetse2020improved}).

\subsection{Annotation Quality of Data Providers}\label{exp:data_prov_quality}

As previously discussed, annotations (in soccer) are frequently obtained from data providers that are not bound to fulfill any requirements or to meet a common gold standard.
To this end, we explore the quality of a data provider on the exemplary \acrfull{dataset_provider} which contains four matches of a first European soccer league from the 2014/2015 season. Here, we avoid an examination of semantically complex events like \textit{successful interference} % where differences in the definition may constrain the obtained results.
and perform an examination of the \textit{successful pass}, \textit{shot}, and \textit{game status changing events} where we find the largest compliance with the data-provider event catalog ~\cite{liu2013reliability}. To obtain a reference, we instruct a domain expert to acquire frame-wise accurate annotations by watching unedited recordings of the matches. 
Similar to the previous experiments, we compute precision and recall while we account for differences in the number of total annotated events by application of \acrshort{scm}~(with specific consideration of passing player identities for passes). The results are given in Table~\ref{tab:aggreement_nn_cm} and a representative example is displayed in Figure~\ref{fig:example_annotations}.
We observe a low agreement between the precise expert and the data-provider annotation~(compared to results for~\acrshort{dataset_soccer}). While due to the consideration of player identities, slightly more \textit{successful pass} events are consistent, the agreement for \acrshort{scm} is also poor.
This is caused by a general imprecision in the data-provider annotation. This imprecision likely originates from the real-time manual annotation which data providers demand. The human annotators are instructed to collect temporal (and spatial) characteristics of specific events while simultaneously deciding on a corresponding event type from a rather large range of \textit{high-level} event catalog ~\cite{liu2013reliability}. 
These results reveal the need for exact definitions and annotation guidelines and emphasize the value of automatic solutions.
We intend to show with this exploratory experiment that, the quality of the annotations provided should be taken into account depending on the targeted application. 
Of course, we cannot draw conclusions about quality of other data and seasons based on this case study.

\section{Conclusions}\label{sec:conclusion}
In this paper, we have addressed the real-world task of fine-grained event detection and spotting in invasion games. 
While prior work already presented automatic methods for the detection of individual sport-specific events with focus on soccer, they lacked objective event definitions and complete description for invasion games.
Despite the wide range of examined events, their complexity, and ambiguity, the data quality had not been investigated making the assessment of automatic approaches difficult. Even current evaluation metrics are inconsistent. 
Therefore, we have contributed a \textit{hierarchical} taxonomy that enables a \textit{minimal} and \textit{objective} annotation and is \textit{modular expendable} to fit the needs of various invasion games. 
In addition, we released two multimodal datasets with gold standard event annotations~(soccer and handball).
Extensive evaluation have validated the taxonomy as well as the quality of our two benchmark datasets while a comparison with data-provider annotations revealed advantages in annotation quality.
The results have shown that high agreement can be achieved even without domain knowledge. In addition, the \textit{hierarchical} approach demonstrates that (semantically) complex events can be propagated to a shared parent event to reach an increase in agreement.

With the presented taxonomy, datasets, and baseline, we create a foundation for the design and the benchmarking of upcoming automatic approaches for the spotting of on-ball events.
Also, other domains that work with video, positional, and event data, could benefit from the taxonomy and the datasets introduced in this paper.
In the future, we plan to integrate non-on-ball events into the taxonomy and to exploit \textit{hierarchical} information and attention to the ball position during training of a deep model.

\section*{Acknowledgement}
This project has received funding from the German Federal Ministry of Education and Research (BMBF -- Bundesministerium für Bildung und Forschung) under 01IS20021A, 01IS20021B, and 01IS20021C.
This research was supported by a grant from the German Research Council (DFG, Deutsche Forschungsgemeinschaft) to DM (grant ME~2678/30.1).

\end{document}